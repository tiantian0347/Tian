% !Mode:: "TeX:UTF-8"
\documentclass[12pt,a4paper]{article}
\input{../LaTeX/模板/en_preamble.tex}
\input{../LaTeX/模板/xecjk_preamble.tex}
\input{../LaTeX/模板/listings.tex}

\title{显式欧拉方法}
\author{作者:田甜}
\date{\chntoday}
\begin{document}
\maketitle
\newpage
\section{显式欧拉方法}
欧拉方法,是一种一阶数值方法,用以对给定初值的常微分方程(即初值問題)求解。它是一种解决数值常微分方程的最基本的一类显型方法。

思想:采用离散变量法,即把一个连续型问题转化成一个离散型问题,即采用步进的方式求出方程在一些离散点上的近似值。

\begin{equation}
\left\{\begin{array}{l}{y^{\prime}=g(x, y)} \\ {y\left(x_{0}\right)=y_{0}}\end{array}\right.
\end{equation}
\textcolor{blue}{推导过程:}
\begin{enumerate}[(1)]
	\item 消除导数项,用差商近似代替导数,将变量离散化。任取一点,设为$x_i$,则有:\begin{equation}
	y^{\prime}\left(x_{i}\right)=g\left(x_{i}, y\left(x_{i}\right)\right)
	\end{equation}
	\item 用差商公式$y^{\prime}\left(x_{i}\right)=\frac{y\left(x_{i+1}\right)-y\left(x_{i}\right)}{x_{i+1}-x_{i}}$替代导数项,假设步长为$h$,得:\begin{equation}
	y\left(x_{i+1}\right) \approx y\left(x_{i}\right)+g\left(x_{i}, y\left(x_{i}\right)\right)
	\end{equation}
	\item 用$y_i$表示$y(x_i)$得近似值,代入上式,得:\begin{equation}
	y_{i+1}=y_{i}+h g\left(x_{i}, y_{i}\right) \quad(i=0,1,2, \ldots)
	\end{equation}
\end{enumerate}
\textcolor{blue}{例:}\begin{equation}
\left\{\begin{array}{l}{y^{\prime}=y} \\ {y\left(x_{0}\right)=y_{0}}\end{array}\right.\end{equation}
\textcolor{blue}{Python代码}
\lstset{language=Python}
\begin{lstlisting}
import numpy as np
import matplotlib.pyplot as plt
a=0
b=1
ya=1
step=200
y=np.zeros([step+1])
x=np.linspace(a,b,step+1)
y[0]=ya
u=np.exp(x)
for i in range(0,step):
  y[i+1]=y[i]+((b-a)/step)*y[i]
  print(y[i])
plt.xlabel("x")
plt.ylabel("error")
plt.plot(x,y-u) 
plt.show()
\end{lstlisting}
计算可得误差图:
\begin{figure}[H]
	\centering
	\includegraphics[scale=0.5]{./figures/Figure_1.png}
\end{figure}

根据上图我们可以发现,误差较小,模型有效。
















































\end{document}
