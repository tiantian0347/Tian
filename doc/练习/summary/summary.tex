% !Mode:: "TeX:UTF-8"
\documentclass[12pt,a4paper]{article}
\input{en_preamble.tex}
\input{xecjk_preamble.tex}

\title{数值试验报告}
\author{作者:田甜}
\date{\chntoday}
\begin{document}
\maketitle
\newpage
\section{编程问题}
在之前几次的编程中,只是简单的写出了数学求解过程,因此在编程时导致赋值时产生混乱。因此在最后一次寻找问题时,将求解过程完完全全的写了下来,而后在进行问题查找时变得十分快捷。在接下来的编程中,我将吸取教训整理出完整的数学算法后在进行编程。

其次,在最初进行编程时,虽然数学求解过程已经明了,但是将数学解法转换成算法思想时总会遇到一些困难,有时会感到无从下手。对于这一点,暂时没有找到特别有效的解决方法。

进行$for$循环时,在最开始没有找到循环的规律,因此一直再用笨方法数循环的范围,尤其是在双层循环时会产生一些迷茫。但写了几次循环后就找到了它的规律。

之前编写的程序中,图片部分一直没有处理好,在请教师兄后发现是调用图片命令的问题,在曲面作图时不应该继续调用简单的$plt$命令。
\section{结果分析}
在五点差分问题求解时,主要问题出现在程序的错误上,没有准确的赋值。

而在最后的误差阶计算时一直没有得到正确的阶数,后来与大家讨论后发现问题的产生是由于在进行二范数求误差时没有除以$n^2$即离散后的未知点的数量。

\section{2019.10.12}
今天在进行编程时,在函数中将$x$的值赋予了$v$,而后再使用$x$进行计算时发现$x$的值发生了改变,在与同门探讨后发现\textcolor{blue}{进行赋值时如果直接使用$“=”$被赋值的变量改变后,之前的变量也相应改变,可以使用copy语句进行解决。}

\end{document}
