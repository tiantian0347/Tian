% !Mode:: "TeX:UTF-8"
\documentclass[12pt,a4paper]{article}

%%%%%%%%------------------------------------------------------------------------
%%%% 日常所用宏包

%% 控制页边距
% 如果是beamer文档类, 则不用geometry
\makeatletter
\@ifclassloaded{beamer}{}{\usepackage[top=2.5cm, bottom=2.5cm, left=2.5cm, right=2.5cm]{geometry}}
\makeatother

%% 控制项目列表
\usepackage{enumerate}

%% 多栏显示
\usepackage{multicol}

%% 算法环境
\usepackage{algorithm}  
\usepackage{algorithmic} 
\usepackage{float} 

%% 网址引用
\usepackage{url}

%% 控制矩阵行距
\renewcommand\arraystretch{1.4}

%% hyperref宏包,生成可定位点击的超链接,并且会生成pdf书签
\makeatletter
\@ifclassloaded{beamer}{
\usepackage{hyperref}
\usepackage{ragged2e} % 对齐
}{
\usepackage[%
    pdfstartview=FitH,%
    CJKbookmarks=true,%
    bookmarks=true,%
    bookmarksnumbered=true,%
    bookmarksopen=true,%
    colorlinks=true,%
    citecolor=blue,%
    linkcolor=blue,%
    anchorcolor=green,%
    urlcolor=blue%
]{hyperref}
}
\makeatother



\makeatletter % 如果是 beamer 不需要下面两个包
\@ifclassloaded{beamer}{
\mode<presentation>
{
} 
}{
%% 控制标题
\usepackage{titlesec}
%% 控制目录
\usepackage{titletoc}
}
\makeatother

%% 控制表格样式
\usepackage{booktabs}

%% 控制字体大小
\usepackage{type1cm}

%% 首行缩进,用\noindent取消某段缩进
\usepackage{indentfirst}

%% 支持彩色文本、底色、文本框等
\usepackage{color,xcolor}

%% AMS LaTeX宏包: http://zzg34b.w3.c361.com/package/maths.htm#amssymb
\usepackage{amsmath,amssymb}
%% 多个图形并排
\usepackage{subfig}
%%%% 基本插图方法
%% 图形宏包
\usepackage{graphicx}
\newcommand{\red}[1]{\textcolor{red}{#1}}
\newcommand{\blue}[1]{\structure{#1}}
\newcommand{\brown}[1]{\textcolor{brown}{#1}}
\newcommand{\green}[1]{\textcolor{green}{#1}}


%%%% 基本插图方法结束

%%%% pgf/tikz绘图宏包设置
\usepackage{pgf,tikz}
\usetikzlibrary{shapes,automata,snakes,backgrounds,arrows}
\usetikzlibrary{mindmap}
%% 可以直接在latex文档中使用graphviz/dot语言,
%% 也可以用dot2tex工具将dot文件转换成tex文件再include进来
%% \usepackage[shell,pgf,outputdir={docgraphs/}]{dot2texi}
%%%% pgf/tikz设置结束


\makeatletter % 如果是 beamer 不需要下面两个包
\@ifclassloaded{beamer}{

}{
%%%% fancyhdr设置页眉页脚
%% 页眉页脚宏包
\usepackage{fancyhdr}
%% 页眉页脚风格
\pagestyle{plain}
}

%% 有时会出现\headheight too small的warning
\setlength{\headheight}{15pt}

%% 清空当前页眉页脚的默认设置
%\fancyhf{}
%%%% fancyhdr设置结束


\makeatletter % 对 beamer 要重新设置
\@ifclassloaded{beamer}{

}{
%%%% 设置listings宏包用来粘贴源代码
%% 方便粘贴源代码,部分代码高亮功能
\usepackage{listings}

%% 设置listings宏包的一些全局样式
%% 参考http://hi.baidu.com/shawpinlee/blog/item/9ec431cbae28e41cbe09e6e4.html
\lstset{
showstringspaces=false,              %% 设定是否显示代码之间的空格符号
numbers=left,                        %% 在左边显示行号
numberstyle=\tiny,                   %% 设定行号字体的大小
basicstyle=\footnotesize,                    %% 设定字体大小\tiny, \small, \Large等等
keywordstyle=\color{blue!70}, commentstyle=\color{red!50!green!50!blue!50},
                                     %% 关键字高亮
frame=shadowbox,                     %% 给代码加框
rulesepcolor=\color{red!20!green!20!blue!20},
escapechar=`,                        %% 中文逃逸字符,用于中英混排
xleftmargin=2em,xrightmargin=2em, aboveskip=1em,
breaklines,                          %% 这条命令可以让LaTeX自动将长的代码行换行排版
extendedchars=false                  %% 这一条命令可以解决代码跨页时,章节标题,页眉等汉字不显示的问题
}}
\makeatother
%%%% listings宏包设置结束


%%%% 附录设置
\makeatletter % 对 beamer 要重新设置
\@ifclassloaded{beamer}{

}{
\usepackage[title,titletoc,header]{appendix}
}
\makeatother
%%%% 附录设置结束


%%%% 日常宏包设置结束
%%%%%%%%------------------------------------------------------------------------


%%%%%%%%------------------------------------------------------------------------
%%%% 英文字体设置结束
%% 这里可以加入自己的英文字体设置
%%%%%%%%------------------------------------------------------------------------

%%%%%%%%------------------------------------------------------------------------
%%%% 设置常用字体字号,与MS Word相对应

%% 一号, 1.4倍行距
\newcommand{\yihao}{\fontsize{26pt}{36pt}\selectfont}
%% 二号, 1.25倍行距
\newcommand{\erhao}{\fontsize{22pt}{28pt}\selectfont}
%% 小二, 单倍行距
\newcommand{\xiaoer}{\fontsize{18pt}{18pt}\selectfont}
%% 三号, 1.5倍行距
\newcommand{\sanhao}{\fontsize{16pt}{24pt}\selectfont}
%% 小三, 1.5倍行距
\newcommand{\xiaosan}{\fontsize{15pt}{22pt}\selectfont}
%% 四号, 1.5倍行距
\newcommand{\sihao}{\fontsize{14pt}{21pt}\selectfont}
%% 半四, 1.5倍行距
\newcommand{\bansi}{\fontsize{13pt}{19.5pt}\selectfont}
%% 小四, 1.5倍行距
\newcommand{\xiaosi}{\fontsize{12pt}{18pt}\selectfont}
%% 大五, 单倍行距
\newcommand{\dawu}{\fontsize{11pt}{11pt}\selectfont}
%% 五号, 单倍行距
\newcommand{\wuhao}{\fontsize{10.5pt}{10.5pt}\selectfont}
%%%%%%%%------------------------------------------------------------------------


%% 设定段间距
\setlength{\parskip}{0.5\baselineskip}

%% 设定行距
\linespread{1}


%% 设定正文字体大小
% \renewcommand{\normalsize}{\sihao}

%制作水印
\RequirePackage{draftcopy}
\draftcopyName{XTUMESH}{100}
\draftcopySetGrey{0.90}
\draftcopyPageTransform{40 rotate}
\draftcopyPageX{350}
\draftcopyPageY{80}

%%%% 个性设置结束
%%%%%%%%------------------------------------------------------------------------


%%%%%%%%------------------------------------------------------------------------
%%%% bibtex设置

%% 设定参考文献显示风格
% 下面是几种常见的样式
% * plain: 按字母的顺序排列,比较次序为作者、年度和标题
% * unsrt: 样式同plain,只是按照引用的先后排序
% * alpha: 用作者名首字母+年份后两位作标号,以字母顺序排序
% * abbrv: 类似plain,将月份全拼改为缩写,更显紧凑
% * apalike: 美国心理学学会期刊样式, 引用样式 [Tailper and Zang, 2006]

\makeatletter
\@ifclassloaded{beamer}{
\bibliographystyle{apalike}
}{
\bibliographystyle{unsrt}
}
\makeatother


%%%% bibtex设置结束
%%%%%%%%------------------------------------------------------------------------

%%%%%%%%------------------------------------------------------------------------
%%%% xeCJK相关宏包

\usepackage{xltxtra,fontspec,xunicode}
\usepackage[slantfont, boldfont]{xeCJK} 

\setlength{\parindent}{2em}%中文缩进两个汉字位

%% 针对中文进行断行
\XeTeXlinebreaklocale "zh"             

%% 给予TeX断行一定自由度
\XeTeXlinebreakskip = 0pt plus 1pt minus 0.1pt

%%%% xeCJK设置结束                                       
%%%%%%%%------------------------------------------------------------------------

%%%%%%%%------------------------------------------------------------------------
%%%% xeCJK字体设置

%% 设置中文标点样式,支持quanjiao、banjiao、kaiming等多种方式
\punctstyle{kaiming}                                        
                                                     
%% 设置缺省中文字体
%\setCJKmainfont[BoldFont={Adobe Heiti Std}, ItalicFont={Adobe Kaiti Std}]{Adobe Song Std}   
\setCJKmainfont{SimSun}
%% 设置中文无衬线字体
%\setCJKsansfont[BoldFont={Adobe Heiti Std}]{Adobe Kaiti Std}  
%% 设置等宽字体
%\setCJKmonofont{Adobe Heiti Std}                            

%% 英文衬线字体
\setmainfont{DejaVu Serif}                                  
%% 英文等宽字体
\setmonofont{DejaVu Sans Mono}                              
%% 英文无衬线字体
\setsansfont{DejaVu Sans}                                   

%% 定义新字体
\setCJKfamilyfont{song}{Adobe Song Std}                     
\setCJKfamilyfont{kai}{Adobe Kaiti Std}
\setCJKfamilyfont{hei}{Adobe Heiti Std}
\setCJKfamilyfont{fangsong}{Adobe Fangsong Std}
\setCJKfamilyfont{lisu}{LiSu}
\setCJKfamilyfont{youyuan}{YouYuan}

%% 自定义宋体
\newcommand{\song}{\CJKfamily{song}}                       
%% 自定义楷体
\newcommand{\kai}{\CJKfamily{kai}}                         
%% 自定义黑体
\newcommand{\hei}{\CJKfamily{hei}}                         
%% 自定义仿宋体
\newcommand{\fangsong}{\CJKfamily{fangsong}}               
%% 自定义隶书
\newcommand{\lisu}{\CJKfamily{lisu}}                       
%% 自定义幼圆
\newcommand{\youyuan}{\CJKfamily{youyuan}}                 

%%%% xeCJK字体设置结束
%%%%%%%%------------------------------------------------------------------------

%%%%%%%%------------------------------------------------------------------------
%%%% 一些关于中文文档的重定义
\newcommand{\chntoday}{\number\year\,年\,\number\month\,月\,\number\day\,日}
%% 数学公式定理的重定义

%% 中文破折号,据说来自清华模板
\newcommand{\pozhehao}{\kern0.3ex\rule[0.8ex]{2em}{0.1ex}\kern0.3ex}

\newtheorem{example}{例}                                   
\newtheorem{theorem}{定理}[section]                         
\newtheorem{definition}{定义}
\newtheorem{axiom}{公理}
\newtheorem{property}{性质}
\newtheorem{proposition}{命题}
\newtheorem{lemma}{引理}
\newtheorem{corollary}{推论}
\newtheorem{remark}{注解}
\newtheorem{condition}{条件}
\newtheorem{conclusion}{结论}
\newtheorem{assumption}{假设}

\makeatletter %
\@ifclassloaded{beamer}{

}{
%% 章节等名称重定义
\renewcommand{\contentsname}{目录}     
\renewcommand{\indexname}{索引}
\renewcommand{\listfigurename}{插图目录}
\renewcommand{\listtablename}{表格目录}
\renewcommand{\appendixname}{附录}
\renewcommand{\appendixpagename}{附录}
\renewcommand{\appendixtocname}{附录}
%% 设置chapter、section与subsection的格式
\titleformat{\chapter}{\centering\huge}{第\thechapter{}章}{1em}{\textbf}
\titleformat{\section}{\centering\sihao}{\thesection}{1em}{\textbf}
\titleformat{\subsection}{\xiaosi}{\thesubsection}{1em}{\textbf}
\titleformat{\subsubsection}{\xiaosi}{\thesubsubsection}{1em}{\textbf}

\@ifclassloaded{book}{

}{
\renewcommand{\abstractname}{摘要}
}
}
\makeatother

\renewcommand{\figurename}{图}
\renewcommand{\tablename}{表}

\makeatletter
\@ifclassloaded{book}{
\renewcommand{\bibname}{参考文献}
}{
\renewcommand{\refname}{参考文献} 
}
\makeatother

\floatname{algorithm}{算法}
\renewcommand{\algorithmicrequire}{\textbf{输入:}}
\renewcommand{\algorithmicensure}{\textbf{输出:}}

%%%% 中文重定义结束
%%%%%%%%------------------------------------------------------------------------


\title{重心坐标函数}
\author{姓名:田甜}
\date{\chntoday}
\begin{document}
\maketitle
\newpage
\section{一维重心坐标函数}
在一维情况下,我们选取两个点$A:x_0,B:x_1(x_0 \neq x_1)$设他们满足方程\begin{equation}
f_i(x_j)=\begin{cases}1 \quad i=j
\\ 0\quad i\neq j\end{cases}\qquad i,j=0,1
\label{eqaution1}
\end{equation}
\subsection{方法一}
此时我们假设\begin{equation}
f_i(x)=a_ix+b_i
\label{eqaution2}
\end{equation}
求解得$$\begin{cases}
\begin{aligned}a_0&=\cfrac{1}{x_0-x_1}\\  b_0&=-\cfrac{x_1}{x_0-x_1}\end{aligned}
\end{cases}
\begin{cases}\begin{aligned}a_1&=\cfrac{1}{x_1-x_0}\\  b_1&=-\cfrac{x_0}{x_1-x_0}\end{aligned}\end{cases}$$
此时就得到了两个基函数,且它们的梯度分别为$$\nabla f_0=a_0,\quad \nabla f_1=a_1$$
\subsection{方法二}
利用长度法取一点$M:x$,这时我们可以得到基函数
\begin{equation}
f_0(x)=\left| \cfrac{x-x_1}{x_0-x_1}\right|
\end{equation}
\begin{equation}
f_1(x)=\left| \cfrac{x-x_0}{x_0-x_1}\right|
\end{equation}
同时在$AB$上运动时函数变化最快,因此得到他们的梯度为$$\nabla f_0=\cfrac{1}{x_0-x_1},\quad \nabla f_1=\cfrac{1}{x_1-x_0}$$
\section{二维重心坐标函数}
在二维情况下,我们选取三个点$A:(x_0,y_0),B(x_1,y_1),C(x_2,y_2)(A,B,C$不共线)。设它们满足方程\begin{equation}
f_i(x_j,y_j)=\begin{cases}1 \quad i=j
\\ 0\quad i\neq j\end{cases}\qquad i,j=0,1,2
\end{equation}
\subsection{方法一}
我们假设\begin{equation}
f_i(x,y)=a_ix+b_iy+c_i
\end{equation}
求解可得$$\begin{cases}\begin{aligned}a_0&=\cfrac{y_2-y_1}{(y_0-y_1)(x_1-x_2)-(y_1-y_2)(x_0-x_1)}\\ b_0&=\cfrac{x_1-x_2}{(y_0-y_1)(x_1-x_2)-(y_1-y_2)(x_0-x_1)}\\
c_0&=\cfrac{x_2y_1-x_1y_2}{(y_0-y_1)(x_1-x_2)-(y_1-y_2)(x_0-x_1)}
\end{aligned}\end{cases}$$
$$\begin{cases}\begin{aligned}a_1&=\cfrac{y_2-y_0}{(y_1-y_2)(x_0-x_2)-(y_0-y_2)(x_1-x_2)}\\ b_1&=\cfrac{x_0-x_2}{(y_1-y_2)(x_0-x_2)-(y_0-y_2)(x_1-x_2)}\\
c_1&=\cfrac{x_0y_2-x_2y_0}{(y_1-y_2)(x_0-x_2)-(y_0-y_2)(x_1-x_2)}
\end{aligned}\end{cases}$$
$$\begin{cases}\begin{aligned}a_2&=\cfrac{y_1-y_0}{(y_2-y_1)(x_0-x_1)-(y_0-y_1)(x_2-x_1)}\\ b_2&=\cfrac{x_0-x_1}{(y_2-y_1)(x_0-x_1)-(y_0-y_1)(x_2-x_1)}\\
c_2&=\cfrac{x_0y_1-x_1y_0}{(y_2-y_1)(x_0-x_1)-(y_0-y_1)(x_2-x_1)}
\end{aligned}\end{cases}$$
此时就得到了三个基函数,且它们的梯度分别为$$\nabla f_0=(a_0,b_0),\quad \nabla f_1=(a_1,b_1),\quad \nabla f_2=(a_2,b_2)$$
\subsection{方法二}
利用面积法,我们取点$M:(x,y)$,且设$\triangle ABC$为逆时针顺序排列,此时计算$\triangle ABC,\triangle BCM,\triangle AMC,\triangle ABM$的面积$S,S_0,S_1,S_2$有:
\begin{equation*}\begin{aligned}
S&=\left| \frac{1}{2} \begin{vmatrix}
x_0&y_0&1\\x_1&y_1&1\\x_2&y_2&1\end{vmatrix}\right|,S_0&=\left| \frac{1}{2} \begin{vmatrix}
x&y&1\\x_1&y_1&1\\x_2&y_2&1\end{vmatrix}\right|\\
S_1&=\left| \frac{1}{2} \begin{vmatrix}
x_0&y_0&1\\x&y&1\\x_2&y_2&1\end{vmatrix}\right|,S_2&=\left| \frac{1}{2} \begin{vmatrix}
x_0&y_0&1\\x_1&y_1&1\\x&y&1\end{vmatrix}\right|
\end{aligned}
\end{equation*}
这时可以得到三个基函数为
\begin{equation}
f_i(x,y)=\frac{s_i}{s},\quad i=0,1,2
\end{equation}

此时基函数$f_i(x,y)$的梯度方向即为$S_i$变化最快的方向,因此基函数$f_0(x,y)$的梯度方向为向$BC$边作垂线,且箭头指向点$A$的方向。即$\overrightarrow {B C}$向左旋转$90^\circ$的方向,得到$$\nabla f_0=\frac{1}{2S}(y_2-y_1,x_1-x_2)$$
同理,得到$f_1(x,y),f_2(x,y)$的梯度为$$\nabla f_1=\frac{1}{2S}(y_2-y_0,x_0-x_2),\nabla f_2=\frac{1}{2S}(y_1-y_0,x_0-x_1)$$
\section{三维重心坐标函数}
\subsection{方法一}
在三维情况下,我们选取四个点$A:(x_0,y_0,z_0),B(x_1,y_1,z_1),C(x_2,y_2,z_2),D(x_3,y_3,z_3)(A,B,C,D$不共面)。设它们满足方程\begin{equation}
f_i(x_j,y_j,z_j)=\begin{cases}1 \quad i=j
\\ 0\quad i\neq j\end{cases}\qquad i,j=0,1,2,3
\label{equation3.1}
\end{equation}
我们假设\begin{equation}
f_i(x,y,z)=a_ix+b_iy+c_iz+d_i
\label{equation3.2}
\end{equation}
此时通过求解公式\ref{equation3.1},\ref{equation3.2}就得到了三个基函数,且得出它们的梯度分别为$$\nabla f_0=(a_0,b_0,c_0),\quad \nabla f_1=(a_1,b_1,c_1),\quad \nabla f_2=(a_2,b_2,c_2)$$
\subsection{方法二}
利用体积法,我们选取点$M(x,y,z)$,此时计算五个四面体$ABCD,MBCD,AMCD,ABMD,ABCM$的体积$V,V_0,V_1,V_2,V_3,V_4$:
$$V=\left| \frac{1}{6}\begin{vmatrix}
x_0&y_0&z_0&1\\x_1&y_1&z_1&1\\x_2&y_2&z_2&1\\x_3&y_3&z_3&1
\end{vmatrix}\right| $$
$$\begin{aligned}V_0&=\left| \frac{1}{6}\begin{vmatrix}
x&y&z&1\\x_1&y_1&z_1&1\\x_2&y_2&z_2&1\\x_3&y_3&z_3&1
\end{vmatrix}\right|,V_1&=\left| \frac{1}{6}\begin{vmatrix}
x_0&y_0&z_0&1\\x&y&z&1\\x_2&y_2&z_2&1\\x_3&y_3&z_3&1
\end{vmatrix}\right|  \\
V_2&=\left| \frac{1}{6}\begin{vmatrix}
x_0&y_0&z_0&1\\x_1&y_1&z_1&1\\x&y&z&1\\x_3&y_3&z_3&1
\end{vmatrix}\right|,V_3&=\left| \frac{1}{6}\begin{vmatrix}
x_0&y_0&z_0&1\\x_1&y_1&z_1&1\\x_2&y_2&z_2&1\\x&y&z&1
\end{vmatrix}\right|   \\
\end{aligned}$$
这时我们得到四个基函数为\begin{equation}
f_i(x,y,z)=\cfrac{V_i}{V},i=0,1,2,3
\label{equation3.3}
\end{equation}

以$f_0(x,y,z)$进行梯度的求解,首先确定梯度方向,过$A$垂直于$\triangle BCD$的垂线,垂足为$P$,此时$\overrightarrow {P A}$的方向即为梯度方向。设$\overrightarrow {P A}=(a,b,c)$则$$\begin{cases}\overrightarrow {P A} \cdot \overrightarrow {B C}=0\\ \overrightarrow {P A} \cdot \overrightarrow {C D}=0\\ \overrightarrow {P A} \cdot \overrightarrow {D B}=0
\end{cases}$$
即\begin{equation}
\begin{pmatrix}
x_2-x_1&y_2-y_1&z_2-z_1\\
x_3-x_2&y_3-y_2&z_3-z_2\\
x_1-x_3&y_1-y_3&z_1-z_3
\end{pmatrix}
\begin{pmatrix}
a\\b\\c
\end{pmatrix}=0
\end{equation}

由假设知$(a,b,c)$有唯一解,将其单位化后得到$\overrightarrow {n_0}=\cfrac {1}{\sqrt {a^2+b^2+c^2}}(a,b,c)$

现在已经确定梯度的方向,结合函数$\ref{equation3.3}$我们可以得出$f_0(x,y,z)$的梯度:$$\nabla f_0(x,y,z)=(\cfrac {a}{\sqrt {a^2+b^2+c^2}} \cfrac{\left|\begin{vmatrix}
	y_1&z_1&1\\y_2&z_2&1\\y_3&z_3&1
	\end{vmatrix}\right|}{6V},
\cfrac {b}{\sqrt {a^2+b^2+c^2}} \cfrac{\left|\begin{vmatrix}
	x_1&z_1&1\\x_2&z_2&1\\x_3&z_3&1
	\end{vmatrix}\right|}{6V},\cfrac {c}{\sqrt {a^2+b^2+c^2}} \cfrac{\left|\begin{vmatrix}
	x_1&y_1&1\\x_2&y_2&1\\x_3&y_3&1
	\end{vmatrix}\right|}{6V})$$

同理,我们可以得到$\nabla f_1(x,y,z),\nabla f_2(x,y,z)$。


































\end{document}
