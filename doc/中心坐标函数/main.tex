% !Mode:: "TeX:UTF-8"
\documentclass[12pt,a4paper]{article}
\input{../LaTeX/模板/en_preamble.tex}
\input{../LaTeX/模板/xecjk_preamble.tex}

\title{重心坐标函数}
\author{姓名:田甜}
\date{\chntoday}
\begin{document}
\maketitle
\newpage
\section{一维重心坐标函数}
在一维情况下,我们选取两个点$A:x_0,B:x_1(x_0 \neq x_1)$设他们满足方程\begin{equation}
f_i(x_j)=\begin{cases}1 \quad i=j
\\ 0\quad i\neq j\end{cases}\qquad i,j=0,1
\label{eqaution1}
\end{equation}
\subsection{方法一}
此时我们假设\begin{equation}
f_i(x)=a_ix+b_i
\label{eqaution2}
\end{equation}
求解得$$\begin{cases}
\begin{aligned}a_0&=\cfrac{1}{x_0-x_1}\\  b_0&=-\cfrac{x_1}{x_0-x_1}\end{aligned}
\end{cases}
\begin{cases}\begin{aligned}a_1&=\cfrac{1}{x_1-x_0}\\  b_1&=-\cfrac{x_0}{x_1-x_0}\end{aligned}\end{cases}$$
此时就得到了两个基函数,且它们的梯度分别为$$\nabla f_0=a_0,\quad \nabla f_1=a_1$$
\subsection{方法二}
利用长度法取一点$M:x$,这时我们可以得到基函数
\begin{equation}
f_0(x)=\left| \cfrac{x-x_1}{x_0-x_1}\right|
\end{equation}
\begin{equation}
f_1(x)=\left| \cfrac{x-x_0}{x_0-x_1}\right|
\end{equation}
同时在$AB$上运动时函数变化最快,因此得到他们的梯度为$$\nabla f_0=\cfrac{1}{x_0-x_1},\quad \nabla f_1=\cfrac{1}{x_1-x_0}$$
\section{二维重心坐标函数}
在二维情况下,我们选取三个点$A:(x_0,y_0),B(x_1,y_1),C(x_2,y_2)(A,B,C$不共线)。设它们满足方程\begin{equation}
f_i(x_j,y_j)=\begin{cases}1 \quad i=j
\\ 0\quad i\neq j\end{cases}\qquad i,j=0,1,2
\end{equation}
\subsection{方法一}
我们假设\begin{equation}
f_i(x,y)=a_ix+b_iy+c_i
\end{equation}
求解可得$$\begin{cases}\begin{aligned}a_0&=\cfrac{y_2-y_1}{(y_0-y_1)(x_1-x_2)-(y_1-y_2)(x_0-x_1)}\\ b_0&=\cfrac{x_1-x_2}{(y_0-y_1)(x_1-x_2)-(y_1-y_2)(x_0-x_1)}\\
c_0&=\cfrac{x_2y_1-x_1y_2}{(y_0-y_1)(x_1-x_2)-(y_1-y_2)(x_0-x_1)}
\end{aligned}\end{cases}$$
$$\begin{cases}\begin{aligned}a_1&=\cfrac{y_2-y_0}{(y_1-y_2)(x_0-x_2)-(y_0-y_2)(x_1-x_2)}\\ b_1&=\cfrac{x_0-x_2}{(y_1-y_2)(x_0-x_2)-(y_0-y_2)(x_1-x_2)}\\
c_1&=\cfrac{x_0y_2-x_2y_0}{(y_1-y_2)(x_0-x_2)-(y_0-y_2)(x_1-x_2)}
\end{aligned}\end{cases}$$
$$\begin{cases}\begin{aligned}a_2&=\cfrac{y_1-y_0}{(y_2-y_1)(x_0-x_1)-(y_0-y_1)(x_2-x_1)}\\ b_2&=\cfrac{x_0-x_1}{(y_2-y_1)(x_0-x_1)-(y_0-y_1)(x_2-x_1)}\\
c_2&=\cfrac{x_0y_1-x_1y_0}{(y_2-y_1)(x_0-x_1)-(y_0-y_1)(x_2-x_1)}
\end{aligned}\end{cases}$$
此时就得到了三个基函数,且它们的梯度分别为$$\nabla f_0=(a_0,b_0),\quad \nabla f_1=(a_1,b_1),\quad \nabla f_2=(a_2,b_2)$$
\subsection{方法二}
利用面积法,我们取点$M:(x,y)$,且设$\triangle ABC$为逆时针顺序排列,此时计算$\triangle ABC,\triangle BCM,\triangle AMC,\triangle ABM$的面积$S,S_0,S_1,S_2$有:
\begin{equation*}\begin{aligned}
S&=\left| \frac{1}{2} \begin{vmatrix}
x_0&y_0&1\\x_1&y_1&1\\x_2&y_2&1\end{vmatrix}\right|,S_0&=\left| \frac{1}{2} \begin{vmatrix}
x&y&1\\x_1&y_1&1\\x_2&y_2&1\end{vmatrix}\right|\\
S_1&=\left| \frac{1}{2} \begin{vmatrix}
x_0&y_0&1\\x&y&1\\x_2&y_2&1\end{vmatrix}\right|,S_2&=\left| \frac{1}{2} \begin{vmatrix}
x_0&y_0&1\\x_1&y_1&1\\x&y&1\end{vmatrix}\right|
\end{aligned}
\end{equation*}
这时可以得到三个基函数为
\begin{equation}
f_i(x,y)=\frac{s_i}{s},\quad i=0,1,2
\end{equation}

此时基函数$f_i(x,y)$的梯度方向即为$S_i$变化最快的方向,因此基函数$f_0(x,y)$的梯度方向为向$BC$边作垂线,且箭头指向点$A$的方向。即$\overrightarrow {B C}$向左旋转$90^\circ$的方向,得到$$\nabla f_0=\frac{1}{2S}(y_2-y_1,x_1-x_2)$$
同理,得到$f_1(x,y),f_2(x,y)$的梯度为$$\nabla f_1=\frac{1}{2S}(y_2-y_0,x_0-x_2),\nabla f_2=\frac{1}{2S}(y_1-y_0,x_0-x_1)$$
\section{三维重心坐标函数}
\subsection{方法一}
在三维情况下,我们选取四个点$A:(x_0,y_0,z_0),B(x_1,y_1,z_1),C(x_2,y_2,z_2),D(x_3,y_3,z_3)(A,B,C,D$不共面)。设它们满足方程\begin{equation}
f_i(x_j,y_j,z_j)=\begin{cases}1 \quad i=j
\\ 0\quad i\neq j\end{cases}\qquad i,j=0,1,2,3
\label{equation3.1}
\end{equation}
我们假设\begin{equation}
f_i(x,y,z)=a_ix+b_iy+c_iz+d_i
\label{equation3.2}
\end{equation}
此时通过求解公式\ref{equation3.1},\ref{equation3.2}就得到了三个基函数,且得出它们的梯度分别为$$\nabla f_0=(a_0,b_0,c_0),\quad \nabla f_1=(a_1,b_1,c_1),\quad \nabla f_2=(a_2,b_2,c_2),\quad \nabla f_3=(a_3,b_3,c_3)$$
\subsection{方法二}
利用体积法,我们选取点$M(x,y,z)$,此时计算五个四面体$ABCD,MBCD,AMCD,ABMD,ABCM$的体积$V,V_0,V_1,V_2,V_3,V_4$:
$$V=\left| \frac{1}{6}\begin{vmatrix}
x_0&y_0&z_0&1\\x_1&y_1&z_1&1\\x_2&y_2&z_2&1\\x_3&y_3&z_3&1
\end{vmatrix}\right| $$
$$\begin{aligned}V_0&=\left| \frac{1}{6}\begin{vmatrix}
x&y&z&1\\x_1&y_1&z_1&1\\x_2&y_2&z_2&1\\x_3&y_3&z_3&1
\end{vmatrix}\right|,V_1&=\left| \frac{1}{6}\begin{vmatrix}
x_0&y_0&z_0&1\\x&y&z&1\\x_2&y_2&z_2&1\\x_3&y_3&z_3&1
\end{vmatrix}\right|  \\
V_2&=\left| \frac{1}{6}\begin{vmatrix}
x_0&y_0&z_0&1\\x_1&y_1&z_1&1\\x&y&z&1\\x_3&y_3&z_3&1
\end{vmatrix}\right|,V_3&=\left| \frac{1}{6}\begin{vmatrix}
x_0&y_0&z_0&1\\x_1&y_1&z_1&1\\x_2&y_2&z_2&1\\x&y&z&1
\end{vmatrix}\right|   \\
\end{aligned}$$
这时我们得到四个基函数为\begin{equation}
f_i(x,y,z)=\cfrac{V_i}{V},i=0,1,2,3
\label{equation3.3}
\end{equation}

以$f_0(x,y,z)$进行梯度的求解,首先确定梯度方向,过$A$垂直于$\triangle BCD$的垂线,垂足为$P$,此时$\overrightarrow {P A}$的方向即为梯度方向。设$\overrightarrow {P A}=(a,b,c)$则$$\begin{cases}\overrightarrow {P A} \cdot \overrightarrow {B C}=0\\ \overrightarrow {P A} \cdot \overrightarrow {C D}=0\\ \overrightarrow {P A} \cdot \overrightarrow {D B}=0
\end{cases}$$
即\begin{equation}
\begin{pmatrix}
x_2-x_1&y_2-y_1&z_2-z_1\\
x_3-x_2&y_3-y_2&z_3-z_2\\
x_1-x_3&y_1-y_3&z_1-z_3
\end{pmatrix}
\begin{pmatrix}
a\\b\\c
\end{pmatrix}=0
\end{equation}
且点$P$在$\triangle BCD$上,即$P(x_0-a,y_0-b,z_0-c)$满足$a(x-x_1)+b(y-y_1)+c(z-z_1)=0$。由此可以得到$(a,b,c)$唯一解,将其单位化后得到$\overrightarrow {n_0}=\cfrac {1}{\sqrt {a^2+b^2+c^2}}(a,b,c)$

现在已经确定梯度的方向,结合函数$\ref{equation3.3}$我们可以得出$f_0(x,y,z)$的梯度:$$\nabla f_0(x,y,z)=(\cfrac {a}{\sqrt {a^2+b^2+c^2}} \cfrac{\left|\begin{vmatrix}
	y_1&z_1&1\\y_2&z_2&1\\y_3&z_3&1
	\end{vmatrix}\right|}{6V},
\cfrac {b}{\sqrt {a^2+b^2+c^2}} \cfrac{\left|\begin{vmatrix}
	x_1&z_1&1\\x_2&z_2&1\\x_3&z_3&1
	\end{vmatrix}\right|}{6V},\cfrac {c}{\sqrt {a^2+b^2+c^2}} \cfrac{\left|\begin{vmatrix}
	x_1&y_1&1\\x_2&y_2&1\\x_3&y_3&1
	\end{vmatrix}\right|}{6V})$$

同理,我们可以得到$\nabla f_1(x,y,z),\nabla f_2(x,y,z),\nabla f_3(x,y,z)$。

\end{document}
