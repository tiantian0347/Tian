% !Mode:: "TeX:UTF-8"
\documentclass[12pt,a4paper]{article}
\input{../LaTeX/模板/en_preamble.tex}
\input{../LaTeX/模板/xecjk_preamble.tex}

\begin{document}
\textcolor{blue}{第三步:}\quad 令$Q_{3}=\operatorname{diag}\left(I_{3 \times 3}, H_{3}\right)$,其中$H_3$是对应于向量$A_2(4:5,3)$的Householder矩阵,则有$$
Q_{3} A_{2}=\left[\begin{array}{ccccc}{*} & {*} &{*} &{*} & {*} \\ {*} & {*} &{*} &{*} & {*} \\ {0} & {*} &{*} & {*} &{*} \\ {0} & {0} & {*} &{*} & {*} \\ {0} & {0} & {0} & {*} &{*}\end{array}\right]
\textbf{和} A_{3} \triangleq Q_{3} A_{2} Q_{3}^{\top}=\left[\begin{array}{ccccc}{*} & {*} &{*} &{*} & {*} \\ {*} & {*} &{*} &{*} & {*} \\ {0} & {*} &{*} &{*} & {*} \\  {0} & {0} & {*} &{*} & {*} \\ {0} & {0} & {0} &{*} & {*}\end{array}\right]
$$
这时,我们就将$A$转化成一个上Hessenberg矩阵,即$QAQ^T=A_3$,其中$Q=Q_3Q_2Q_1$是正交矩阵,$A_3$是上Hessenberg矩阵。

上Hessenberg化算法\\
\textcolor{blue}{算法 5.1}\quad 上Hessenberg化算法(Upper Hessenberg Reduction)
\begin{enumerate}[1:]
	\item set $Q=I$
	\item for $k=1$ to $n-2$ do
	\item \quad compute Hessenberg matrix $H_k$ with respect to $A(k+1:n,k)$
	\item \quad $\begin{aligned} A(k+1& : n, k : n )=H_{k} \cdot A(k+1 : n, k : n) \\ &=A(k+1 : n, k : n)-\beta_{k} v_{k}\left(v_{k}^{\top} A(k+1 : n, k : n)\right) \end{aligned}$
	\item \quad $\begin{aligned} A(1 : n, &k+1 : n)=A(1 : n, k+1 : n) \cdot H_{k}^{\top}\\
	&=A(1 : n, k+1 : n)-\beta_{k} A(1 : n, k+1 : n) v_{k} v_{k}^{\top} \end{aligned}$
	\item \quad $\begin{aligned} Q(k+1& : n, k : n )=H_{k} \cdot Q(k+1 : n, k : n) \\ &=Q(k+1 : n, k : n)-\beta_{k} v_{k}\left(v_{k}^{\top} Q(k+1 : n, k : n)\right) \end{aligned}$
	\item end for
\end{enumerate}

\textcolor{blue}{说明:}
\begin{itemize}
	\item 在实际计算时,我们不需要显式地形成Householder矩阵$H_k$。
	\item 上述算法的运算量大约为$\frac{14}{3} n^{3}+\mathcal{O}\left(n^{2}\right)$。如果不需要计算特征向量,则正交矩阵$Q$也不用计算,此时运算量大约为$\frac{10}{3} n^{3}+\mathcal{O}\left(n^{2}\right)$。
	\item 上Hessenberg矩阵的一个很重要的性质就是在QR迭代中保持形状不变。
\end{itemize}

\textcolor{blue}{定理}\quad 设$A \in \mathbb{R}^{n \times n}$是非奇异上Hessenberg矩阵,其QR分解为$A=QR$,则$\tilde{A} \triangleq R Q$也是上Hessenberg矩阵。

若$A$是奇异的,也可以通过选取适当的$Q$,使得上述结论成立。

由此可知,如果$A$是上Hessenberg矩阵,则QR迭代中的每一个$A_k$都是上Hessenberg矩阵矩阵。这样在进行QR分解时,运算量可大大降低。

Hessenberg矩阵另一重要性质:在QR迭代中保持下次对角线元素非零。

\textcolor{blue}{定理}\quad 设$A \in \mathbb{R}^{n \times n}$是上Hessenberg矩阵且下次对角线元素均非零,即$a_{i+1, i} \neq 0, i=1,2, \ldots, n-1$。设其QR分解为$A=QR$,则$\tilde{A} \triangleq R Q$的下次对角线元素也都非零。

若$A$村咋子某个下次对角线元素为零,则$A$一定可约。因此,我们只需考虑下次对角线均非零的情形。

\textcolor{blue}{推论}\quad $\tilde{A} \triangleq R Q$则在带位移的QR迭代中,所有的$A_k$的下次对角线元素均非零。
\subsubsection{隐式QR迭代}
在QR迭代中,我们要先做QR分解$A_k=Q_kR_k$,然后计算$A_{k+1}k=Q_kR_k$.但事实上,我们可以直接计算出$A_{k+1}$。这就是\textcolor{blue}{隐式QR迭代}。

不失一般性,我们假定$A$是不可约的上Hessenberg矩阵。

隐式QR迭代的理论基础就是下面的\textcolor{blue}{隐式Q定理}。

\textcolor{blue}{定理(ImplicitQTheorem)}\quad 设$H=Q^{\top} A Q \in \mathbb{R}^{n \times n}$是一个不可约上Hessenberg矩阵,其中$Q \in \mathbb{R}^{n \times n}$是正交矩阵,则$Q$的第$2$至第$n$列均由$Q$的第一列所唯一确定(可相差一个符号)。

由于$Q_k$的其他列都由$Q_k$的第一列唯一确定(至多相差一个符号),所以我们只要找到一个正交矩阵$\tilde{Q}_{k}$使得其第一列与$\tilde{Q}_{k}$的第一列相等,且$\tilde{Q}_{k}^{\top} A_{k} \tilde{Q}_{k}$为上Hessenberg矩阵,则由隐式$Q$定理可知$\tilde{Q}_{k}=W Q_{k}$,其中$W=\operatorname{diag}(1, \pm 1, \ldots, \pm 1)$,于是$$
\tilde{Q}_{k}^{\top} A_{k} \tilde{Q}_{k}=W^{\top} Q_{k}^{\top} A_{k} Q_{k} W=W^{\top} A_{k+1} W
$$。
又$W^{\top} A_{k+1} W$与$A_{k+1}$相似,且对角线元素相等,而其他元素也至多相差一个符号,所以不会影响$A_{k+1}$的收敛性,即下三角元素收敛到$0$,对角线元素收敛到$A$的特征值。

在QR迭代算法中,如果我们直接令$A_{k+1}=\tilde{Q}_{k}^{\top} A_{k} \tilde{Q}_{k}$,则其收敛性与原QR迭代算法没有任何区别!这就是隐式QR迭代的基本思想。

由于$A$是上Hessenberg矩阵,因此在实际计算中,我们只需Givens变换。

















































\end{document}
