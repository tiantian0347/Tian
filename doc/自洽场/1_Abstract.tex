%==================================================================================================================================%
%======================================================  摘要 =====================================================================%
%==================================================================================================================================%
\pagestyle{plain}
\pagenumbering{arabic}    %设置页码格式为arabic,并把页计数器置为1.
\chapter*{摘~~~~~~要}
\songti \xiaosihao
分数阶微分方程源于实际。区别于整数阶,分数阶微分方程可以更好地模拟一些动态过程与物理现象,在Engineering Science、Physics、Finance、Hydrology等范畴中起到着重要的作用。大多数分数阶微分方程的解析解有复杂的形式,不便于近似计算,因此,数值求解成了研究分数阶微分方程最重要的方法之一。本文主要研究一维空间分数阶对流-扩散方程的数值解法。通过离散Riemann-Liouville空间分数阶导数,构造Lévy-Feller对流—扩散方程的差分格式,利用有限差分方法求得数值解,并考虑加入干扰项,分析该格式的稳定性。

\vskip 0.4cm
\noindent {\bf 关键词:}Lévy-Feller对流-扩散方程\cite{lqx2007}\cite{Liu2007Approximation};Riemann-Liouville空间分数阶导数;有限差分方法;Grünwald-Letnikov分数阶导数移位算子
%----Abstract------------------------------------------------------------------------------------
\chapter*{\bf Abstract}
\setlength{\baselineskip}{20pt plus2pt minus1pt} % 调整自然行距

Fractional differential equations derive from reality. Different from integral order differential equation, fractional differential equation can better simulate some dynamic processes and physical phenomena, playing an important role in Engineering Science, Physics, Finance, Hydrology and other fields. Most analytic solutions of fractional differential equations have complex forms, which makes it difficult to approximate. Therefore, numerical solution has become one of the most important methods to study fractional differential equations. The numerical solution of fractional convection-diffusion equations in one-dimensional space is studied. The fractional derivative of Riemann-Liouville space is discretized to construct the difference scheme of the Lévy-Feller convection-diffusion equation.


\vskip 0.2cm
\noindent{\bf Key words:} The Lévy-Feller convection-diffusion equation;Riemann-Liouville space fractional derivative;Finite difference method;The Grünwald-Letnikov fractional derivative shift operator
\setlength{\baselineskip}{20pt plus2pt minus1pt} % 调整自然行距
