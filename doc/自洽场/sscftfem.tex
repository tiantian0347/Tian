%%%%%%%%%%%%%%%%%%%%%%%%%%
%      %
% version             %
%%%%%%%%%%%%%%%%%%%%%%%%%%

%\documentclass[aps,floatfix,12pt]{revtex4-1}
%%\documentclass[twocolumn,aps,floatfix]{revtex4}
%
%\usepackage{graphicx,epsfig,subfig,dcolumn,bm,mathrsfs,amsmath,amssymb}
%\usepackage{color}
%\bibliographystyle{apsrev}


\documentclass[final,1p,times]{elsarticle}
%\documentclass[final,3p,times]{elsarticle}
%% \documentclass[final,1p,times,twocolumn]{elsarticle}

\journal{Journal of Computational Physics}

\usepackage{mathrsfs,amsmath,amssymb,bm}
\usepackage{hyperref}
\usepackage{graphicx}
\usepackage{subfigure}
\usepackage{caption}
\usepackage{overpic}
\usepackage{epsfig}
\usepackage{xcolor}
\usepackage{indentfirst}
\usepackage{fancyhdr}
\usepackage{comment}
\usepackage{makecell, rotating}
\usepackage{algorithm} 
\usepackage{algorithmic} 
\usepackage{float}
\usepackage{fullpage} 

\usepackage{xkeyval}
\usepackage{todonotes}
\presetkeys{todonotes}{inline}{} 
\usepackage[draft]{changes}
%\usepackage{lipsum}% <- For dummy text
\definechangesauthor[name={Huayi Wei}, color=red]{why}

\newcommand{\bbR}{\mathbb{R}}
\newcommand{\bbZ}{\mathbb{Z}}
\newcommand{\bbQ}{\mathbb{Q}}
\newcommand{\bh}{\mathbf{h}}
\newcommand{\bx}{\mathbf{x}}
\newcommand{\by}{\mathbf{y}}
\newcommand{\br}{\mathbf{r}}
\newcommand{\bt}{\mathbf{t}}
\newcommand{\bs}{\mathbf{s}}
\newcommand{\bu}{\mathbf{u}}
\newcommand{\bk}{\mathbf{k}}
\newcommand{\bM}{\mathbf{M}}
\newcommand{\bP}{\mathbf{P}}
\newcommand{\bQ}{\mathbf{Q}}
\newcommand{\calS}{\mathcal{S}}
\newcommand{\hF}{\hat{F}}
\newcommand{\bq}{\mathbf{q}}
\newcommand{\hw}{\hat{w}}
\newcommand{\hphi}{\hat{\phi}}
\newcommand{\hxi}{\hat{\xi}}
\newcommand{\hJ}{\hat{J}}
%\def\bx{{\bm x}}

\newtheorem{thm}{Theorem}[section]
\newtheorem{lemma}[thm]{Lemma}
\newtheorem{proposition}[thm]{Proposition}
\newtheorem{corollary}[thm]{Corollary}
\newtheorem{defy}[thm]{Definition}
\newtheorem{remark}[thm]{Remark}
\newtheorem{sample}[thm]{Example}
\newtheorem{assumption}[thm]{Assumption}
\newtheorem{prop}[thm]{Properties}

%\renewcommand{\baselinestretch}{1.5}

\begin{document}

\title{
A finite element method of the self-consistent field theory
on a general curved surface}

\author{Huayi Wei}
\author{Ming Xu}
\author{Kai Jiang\corref{cor}}
\address{
 School of Mathematics and Computational Science, Xiangtan
 University, Xiangtan, Hunan, P.R. China, 411105,
 \\
Hunan Key Laboratory for Computation and Simulation in Science and Engineering
 }
 \cortext[cor]{Corresponding author. Email: kaijiang@xtu.edu.cn.}

\date{\today}

\begin{abstract}
    In this article, we develop a linear surface finite element
	method for the study of self-assembled phases of block
	copolymers on a general curved surface based on the
	self-consistent field theory. Meanwhile, we propose an
    adaptive approach to optimize the size of the general curved surface
    which can capture the compatible surface for a given self-assembled
    pattern.  To demonstrate the power of the approach, we investigate the
    self-assembled patterns of diblock copolymers on several distinct curved
    surfaces, including five closed surfaces and an open surface.  Numerical results
    illustrate the efficiency of our method.  The obtained ordered structures
    are consistent with the previous results on standard surfaces, such as
    sphere and torus. Furthermore, the proposed numerical framework has the
    capability of studying the phase behaviors on arbitrary surfaces precisely. 
\end{abstract}


\maketitle

\section{Introduction}
\label{sec:intrd}

Geometry plays a key role in many scientific
fields\,\cite{nakahara2003topology}, including Hamiltonian mechanics with
constraints, general relativity, quantum mechanics, quantum field theory, the
arrangement of electrons on a sphere (Thomson problem), defect motion in a curved
surface, polymer field theory, generic reaction-diffusion model occurring in
chemistry and biology.  Recent years, the microphase separation of block
copolymer under various types of geometrical confinements both in the bulk
phase and on surfaces has been attracted tremendous attention.  Many novel
patterns of block copolymer emerge from rearrangements of the traditional
ordered phases when adapting to these restrictions\,\cite{wu2004composite,
xiang2005, yu2006prl, charlotte2011interplay}.  The self-assembly of block
copolymers under geometrical constraint provides an efficient means on the
nanometer scale for such applications as the construction of high-capacity data
storage devices, waveguides, quantum dot arrays, dielectric mirrors, nanoporous
membranes, and nanowires, interference
lithography\,\cite{charlotte2011interplay, segalman2005patterning}.  However,
the curvature effects on the self-assembled pattern formation of block
copolymers confined to a curved surface are still far from being fully
understood, although several theoretical works have been devoted to this
problem\,\cite{chantawansri2007, li2014self, li2006self}.  The main obstacle
may be that it still lacks a general theoretical framework which can properly
describe the microphase separation of block copolymers on a general curved
surface.


%The
%curvature of manifold always show itself in
%two sides. In one side, the motion in the
%manifold can be regarded to be the motion in
%high dimensional flat space with a constraint


%Theoretically, the field of lattices constrained on surfaces of
%constant curvature has been covered and explored extensively. The
%main focus remains not only on spherical geometry, with the
%advantage of experimental relevance and well-described
%parameters\,\cite{}, but also on more abstract surfaces of constant
%negative curvature\,\cite{nelson1983liquids}. The problem of
%identifying the ground state at zero temperature has proven to be
%very challenging for large numbers of particles on a sphere and
%is still under investigation. The major complication, from an
%analytical as well as from a simulation standpoint, is the vast
%number of states with very small differences in energy.

Theoretically, self-consistent field theory
(SCFT)\,\cite{fredrickson2006equilibrium} proposes a successful framework in
the study of the self-assembling of block copolymers.  It has predicted a
variety of ordered bulk phases that have been observed in experiments, e.g.,
lamella, hexagonal cylinder, sphere and gyroid order phases of diblock
copolymers\,\cite{matsen1994stable}, and many intricate ordered phases of
triblock copolymers\,\cite{jiang2015self}.  SCFT is a very complicated
variational problem, possessing many unsatisfactory features, such as
saddle-point optimization, nonlinearity, multi-solutions, and multi-parameters.
Analytically solving this issue goes beyond current technologies. An alternating
approach is the numerical technique.  However, it is still a highly challenging
task to extend SCFT from flat spaces to curved surfaces, mostly due to the
inappropriate numerical technique adopted in solving the relevant SCFT
equations.

Numerically solving the SCFT equations requires three parts to study, including
initial values\,\cite{xu2013strategy, jiang2010spectral, jiang2013discovery},
discretization schemes (mainly for the PDE), and
nonlinear iteration methods for seeking the saddle points of the fields.
A key issue of numerically solving SCFT is to discretize the partial
differential equations (PDEs) involved in this model. In flat spaces, there many numerical
approaches have been developed to solve this model  equation.  In general, two
classes of discretization schemes have been presented to discrete the SCFT
equations in the past 30 years.  The first type is the projective-space
discretization method which discretizes equations in a special subspace based
on specific problems.  According to a given pattern and its symmetric group in
microphase-separated block copolymers, the SCFT equations can be expanded in
terms of a set of symmetric basis functions. One of the typical representatives
was the projective Fourier method proposed by Masten and Schick in
1994\,\cite{matsen1994stable}.  The second type is the whole-space
discretization method whose approximated space is the whole space. This method
can be carried out both in real space\,\cite{drolet1999combinatorial} and in
Fourier-space\,\cite{guo2008discovering}.  It has also been demonstrated that
the whole-space discretization methods are able to capture new
patterns\,\cite{drolet1999combinatorial, guo2008discovering}.  In recent years,
the pseudospectral method has been introduced to solve the PDE in
SCFT\,\cite{rasmussen2002improved, cochran2006stability}.  It fully takes
advantage of the best performance of real space and Fourier-space and reduces
the computational complexity to $O(M\log M)$, with the number of degrees $M$,
based on the Fast Fourier Transformation (FFT). 
%Besides space discretization method, 
%second-order operator-splitting method, Crank-Nicokson scheme, and
%linear multi-step approach have been used in the time
%discretization direction. 

From flat spaces to curved surfaces, the Laplace operator in the PDE becomes
the Beltrami-Laplacian operator which is the heat kernel on the Riemann
manifold.  The objective of developed numerical methods is how to discretize
the differential operator on a curved surface.  There are always two viewpoints
to demonstrate the curved surface. The first one is to regard the curved
surface in high dimensional flat space with a constraint and this constraint is
realized through the embedding of this curved surface into high dimensional
flat space. In another viewpoint, the curved surface is presented by the inner
coordinates of the manifold.  From the first viewpoint, the mask method, which
uses a large enough domain in high dimensional flat space to cover the curved
surface, can be chosen to treat the curved surface issue. These methods
developed in the flat space can be used to solve the curved surface
problem\,\cite{li2014mean}. Then the results on the curved surface can be
obtained by restricting the bulk result on the manifold. From another
viewpoint, one shall directly discretize the function on a curved surface. For
the special curved surfaces, such as spherical surface, we can expand the
function defined on the surface by the eigenfunctions of the Beltrami-Laplacian
operator which are the spherical harmonic functions\,\cite{chantawansri2007,
vorselaars2011self}.  However, for a general curved surface, it is impossible
to find out the eigenfunctions to expand the function on the general manifold.
It causes inconvenience in solving SCFT equations on the general curved
surfaces.  In 2014, Li el al.\,\cite{li2014self} mimicked the process of finite
difference method and developed an extended spherical ADI finite
difference\,\cite{li2006self} to solve SCFT on the general curved surfaces. 

%For a given curved surface, their method generates a
%dense enough mesh which is a polyhedron to approximate the
%manifold. Then the ADI finite difference is applied on the
%polyhedron to solve SCFT. 


%A few numerical methods have been
%developed to solve SCFT equations, and mainly concentrated on
%special surfaces, such as spherical surface. 
%In 2006, Li et al.\,\cite{li2006self} designed
%an ADI finite difference scheme and studied
%the microphase patterns of diblock and triblock copolymers on a
%sphere. Later, Chantawansri et
%al.\,\cite{chantawansri2007} solved the diffusion equation on a
%sphere using the pseudo-spectral method with a spherical harmonic
%basis and systematically studied
%the influences of sphere size on the assembled patterns of
%diblock copolymers. Some other numerical schemes have
%also been developed to deal with complex geometries. For
%instance, Kim et al.\,\cite{kim2008interaction} have studied the
%steric interaction between two polymer-grafted particles using SCFT
%with a multiple-coordinate numerical scheme. In order to
%investigate the behaviors of block copolymers grafted to a
%sphere, Vorselaars et al.\cite{vorselaars2011self} developed a
%hybrid numerical
%scheme that combined a spherical-harmonics expansion for
%the angular coordinates with a modified real-space Crank-Nicolson
%method for the radial direction to solve the SCFT equations. 
%For a general curved-surface SCFT, Li et al.\,\cite{li2014self}
%have developed an extended spherical ADI finite difference
%method, and study the phase behaviors of diblock copolymers on a
%quasi-flat surface and general curved surfaces. 
%However there still lack enough efficient numerical methods to
%study the self-assembled behaviors on a general curved-surface
%especially with rigorous theoretical guarantee, such as numerical
%precision, convergent property and so on. 


Fortunately, these methods using local basis functions can be designed to treat
the general curved surface problem.  It is well-known that as an effective
numerical method for solving PDE, the finite element method plays an important
role in modern scientific and engineering computing.  Based on variational
principle, surface finite element method subdivides the definition domain into
small and simple regions, such as triangle, quadrilateral, tetrahedron,  or
hexahedron, etc., and then simple algebraic equations on each small region can
be created. At last these simple equations are assembled into a larger
algebraic system to model the PDE problem.  The advantage of finite element
method is that it can settle complex geometry domains and complex boundaries.
Therefore surface finite element method is a good choice to solving PDEs on a
general curved surface.  The surface finite element methods for PDEs defined on
manifolds have been studied many in the literature; for example see \cite{Dziuk1988,
    Dziuk;Elliott2007, wei2010, Dziuk1991, Dziuk2007, Demlow2009,
Dziuk;Elliott2013}, and reference therein.  The approach was firstly proposed
by Dziuk to solve the Laplace-Beltrami equation on arbitrary surfaces using
linear finite element\,\cite{Dziuk1988}.  Dziuk and Elliott
\cite{Dziuk;Elliott2007} applied the surface linear finite element to parabolic
equations, and gave an example of error bounds in the case of
semi-discretization in space for a fourth order linear problem.  Our previous
work \cite{wei2010} generalized the superconvergence results and several
gradient recovery approaches of linear finite element methods from flat spaces
to general curved surfaces for the Laplace-Beltrami equation with mildly
structured triangular meshes.  In this work, we will continue to develop the
surface finite element method to study the microphase separation of block
copolymer on a general curved surface based on the SCFT.


The size of the curved surface also affects self-assembling patterns.
For example, the self-assembled periodic structures are affected
by the computational domain in flat
space\,\cite{matsen1994stable}.  On the spherical surface, the
radius can affect the number of microdomains of cylindrical phase
as well as the energy values, and phase transition of lamellar
phases.  Hence, our developed method will contain an automatic
mechanism to optimize the size of the general curved surface.  


%In this work, we will develop a finite element approach to
%

%investigate the self-assembly behavior of block copolymers on a
%general curved surface using SCFT. The proposed numerical scheme
%has a rigorous theoretical guarantee, such as convergence,
%precision. Specifically, the proposed finite element method is an
%extension of our previous work for the Laplace-Beltrami operator
%on general curved surfaces\,\cite{wei2010}. At the same time, we
%have noted that the size of the general curved surface will
%affect the assembled phase. 
%Hence, our developed method will 
%contain an automatic mechanism to optimize the size
%of the general curved surface.
The remaining portion is arranged as follows. In Sec.\,\ref{sec:pre}, we
introduce the preliminary knowledge of surface finite element method and
curved surfaces to be used in this paper.  In Sec.\,\ref{sec:scft} we derive the
SCFT on the general curved surface. The numerical algorithm for SCFT coupled
with the surface linear finite element is given in Sec.\,\ref{sec:method}
detailedly.  The efficiency of our method and the self-assembled structures of
diblock copolymers on several curved surfaces are presented in
Sec.\,\ref{sec:rslt}. In the Sec.\,\ref{sec:conclusion}, we will draw the
conclusion and outline the further work.

\section{Preliminaries}\label{sec:pre}

In this section, we will introduce some basic knowledge about
surface finite element method. More details can be found in
\,\cite{Dziuk;Elliott2013}. Meanwhile, the curved surfaces to be used 
in the following calculations and the approach of how to generate
optimal meshes to discrete these surfaces will be presented.

\subsection{Surface finite element method}

Let $\calS$ be a two-dimensional, compact, and $\mathcal C^2$-hypersurface in $\mathbb
R^3$.  Here we assume there exists a smooth level set function $d(\bx)$
that can describe the surface $\calS$. 
\begin{equation*}
    \calS = \{\bx\in U ~|~ d(\bx) = 0\},
\end{equation*}
where $U$ is open subset of $\mathbb R^3$ in which $\nabla d(\bx) \not=0$, and
$d\in \mathcal C^2(U)$. We define the unit normal vector on
$\calS$ by 
\begin{equation*}
    \boldsymbol n = \frac{\nabla d(\bx)}{|\nabla d(\bx)|},
\end{equation*}
which means that the orientation of $\calS$ is fixed through normal
vector $\boldsymbol n$.

For $v(\bx)\in \mathcal C^1(\calS)$, since $\calS$ is $\mathcal C^2$, we can extent $v$ to $\mathcal C^1(U)$
and still denote the extension by $v$. The tangential gradient of
$v$ on $\calS$ can be written as 
\begin{equation*}
    \nabla_S v = \nabla v - (\nabla v\cdot\boldsymbol n)\boldsymbol n =
    \boldsymbol P(\nabla v),
\end{equation*}
where $\boldsymbol P(\bx) = (I - \boldsymbol n\boldsymbol n^t)(\bx)$ is the projection operator
to the tangent plane at a point $\bx\in \calS$. Notice that we use the extension of
$v$ to define the surface gradient. However, it can be shown that $\nabla_S v$ depends only
on the value of $v$ on $\calS$ but not on the
extension\,\cite{Demlow2009}. Namely, $\nabla_S$ is an intrinsic
operator. 

Similarly, for a vector field $\boldsymbol v \in (\mathcal
C^1(\calS))^3$, we can also extend it
to $(\mathcal C^1(U))^3$ and define the tangential divergence operator of
$\boldsymbol v$ on $\calS$ as
\begin{equation*}
    \nabla_S\cdot \boldsymbol v(\bx) = \nabla\cdot\boldsymbol v - \boldsymbol
    n^t\nabla \boldsymbol
    v\nabla\boldsymbol n,
\end{equation*}

The Laplace-Beltrami operator on $\calS$ reads as follows:
\begin{equation*}
    \Delta_\calS v = \nabla_\calS\cdot(\nabla_\calS v) = \Delta v - (\nabla v\cdot\mathbf
    n)(\nabla\cdot\mathbf n) - \mathbf n^t\nabla^2 v \mathbf n,
\end{equation*}
provided $v(\bx) \in \mathcal C^2(\calS)$ and $\nabla^2 v$ is the Hessian matrix of $v$ (suitably
extended as a $C^2(U)$ matrix function). The Sobolev
spaces on surface $\calS$ can be defined us:
\begin{equation*}
    H^1(\calS) = \{v\in L^2(\calS)\, |\, \nabla_S v \in
    (L^2(\calS))^3 \}.
\end{equation*}

Let $\calS$ be approximated by a triangular mesh $\calS_h$ with node set $\mathcal
N_h = \{\bx_i\}_{i=1}^{N}$ and triangle set $\mathcal T_h = \{\tau_h\}$.  We
assume that these triangles are shape-regular and quasi-uniform of a
diameter $h$ and their vertices lie on $\calS$. For any $\tau_h\in \mathcal T_h$,
let $\boldsymbol n_h$ be the unit outward normal vector of $S_h$ on $\tau_h$.
Let $V_h$ be the continuous piecewise linear finite element space on $S_h$,
with linear Lagrangian basis functions $\{\varphi_i(\bx)\}_{i=1}^N$ defined
on $S_h$. $\varphi_i(\bx)$ is the piecewise linear function on each
triangle face $\tau_h$
\begin{equation}\label{eq:varphi}
    \varphi_i(\bx_j) = 
    \begin{cases}
        1,&\text{ if } i=j,\\
        0,&\text{ if } i\not=j
    \end{cases} 
\end{equation}
For $v_h\in C(S_h)$ and $v_h|_{\tau_h}\in C^1(\tau_h)$, we have 
\begin{equation*}
    \nabla_{S_h} v_h|_{\tau_h} := \nabla v_h - (\nabla v_h\cdot\boldsymbol
    n_h)\boldsymbol n_h = (\boldsymbol I - \boldsymbol n_h \boldsymbol
    n_h^t)\nabla v_h = \boldsymbol P_h\nabla v_h.
\end{equation*}


\subsection{Surface representation and mesh generation}

In general, there are three ways to represent a continuous surface in $\mathbb R^3$:
explicit, implicit and parametric. 
In this subsection, we will discuss how to generate good quality
finite element meshes on surfaces with different representations.
A high-quality mesh on a curved surface has these properties:
the triangle elements are all almost equilateral and more mesh
nodes located at the place with big curvature\,\cite{wei2010}.
It will result in better numerical accuracy in solving PDEs. 

For a parabolic surface defined by the explicit function $f(x, y) = x^2 +
y^2$ on a plain domain,  we first generate a good quality triangle mesh on the plain domain, and
then lift the mesh nodes onto the parabolic surface by setting $z
= x^2 +y^2$, as shown in Fig.\,\ref{fig:mesh-Parabolic}. 

\begin{figure}[H]
    \setlength{\captionmargin}{2pt}
    \centering
    \subfigure[]{\includegraphics[scale=0.35]{figures/mesh/circlemesh.pdf}}
    \hspace{1cm}
    \subfigure[]{\includegraphics[scale=0.45]{figures/mesh/parabolic.pdf}}
    \caption{The triangle mesh on the unit circle domain (a) and lift
    into the parabolic surface (b).}
    \label{fig:mesh-Parabolic}
\end{figure}

Torus surface can be defined parametrically by
\begin{align*}
    x(\theta, \phi) &= (R + r\cos\theta)\cos\phi,\\
    y(\theta, \phi) &= (R + r\cos\theta)\sin\phi,\\
    z(\theta, \phi) &= r\sin\theta,
\end{align*}
where $\theta$ and $\phi$ are the angles which make the full circles, so that
their values start and end at the same point, $R$ is the major radius which is
the distance from the center of the tube to the center of the torus, $r$ is the
minor radius which is the radius of the tube. The ratio $R$ divided by $r$ is
known as the "aspect ratio". With above parameter representation,  the triangle
mesh on torus surface can be generated by mapping a triangle mesh on the square
domain $[0, 2\pi]^2$ onto the torus surface, see Fig.\,\ref{fig:mesh-torus}. 
\begin{figure}[H]
    \setlength{\captionmargin}{2pt}
    \centering
    \includegraphics[scale=0.45]{figures/mesh/torus1.pdf}
    \caption{The triangle mesh on torus surface.}
    \label{fig:mesh-torus}
\end{figure}

Any point on the unit sphere
satisfies the following implicit equation
$ \sqrt{x^2 + y^2 + z^2} - 1 = 0$.
A good triangle mesh on spherical surface can be easily
generated by uniformly refining an icosahedron and projecting the new mesh
nodes which are the old mesh edge center onto the sphere surface. 
The refinement procedure is illustrated in
Fig.\,\ref{fig:mesh-sphere}. 

\begin{figure}[H]
    \setlength{\captionmargin}{2pt}
    \centering
    \subfigure[Icosahedron.]{\includegraphics[scale=0.35]{figures/mesh/sphere-init.pdf}}
    \hspace{0.5cm}
    \subfigure[Refine 1]{\includegraphics[scale=0.40]{figures/mesh/sphere-1.pdf}}
    \hspace{0.5cm}
    \subfigure[Refine 2]{\includegraphics[scale=0.38]{figures/mesh/sphere-2.pdf}}
    \hspace{0.5cm}
    \subfigure[Refine 3]{\includegraphics[scale=0.38]{figures/mesh/sphere-3.pdf}}
    \caption{The icosahedron and the uniformly refined mesh on unit sphere.}
    \label{fig:mesh-sphere}
\end{figure}

Three more complex surfaces, including the double torus, heart, and
orthocircle, will be used in the simulations. Their
implicit expressions are given as follows:
\begin{description}
    \item[Double torus surface] 
        \begin{align}
            x^2(x^2-1)[x^2(x^2-1) + 2y^2] + y^4+ z^2 -0.04 =0.
            \label{fig:mesh:doubletorus}
        \end{align}
    \item[Heart surface] 
        \begin{align}
            (x-z^2)^2 +y^2+z^2-1=0.
            \label{fig:mesh:heart}
        \end{align}
    \item[Orthocircle surface]
        \begin{align}
            [(x^2+y^2-1)^2+z^2][(y^2+z^2-1)^2+x^2][(z^2+x^2-1)^2+y^2]-0.075^2[1+3(x^2+y^2+z^2)]=0.
            \label{fig:mesh:orthocircle}
        \end{align}
\end{description}

It is difficult to generated triangle surface mesh by
the method used for the sphere or torus surface. Here we use the 3D Surface Mesh
Generation package in CGAL \cite{rieau2018} to generate the initial triangle
mesh, but its quality is not good enough. So we further optimized these
initial mesh by the method based on the Centroidal Voronoi Tessellation (CVT)
technique \cite{wei2012}, see Fig.\,\ref{fig:mesh-CGAL}. 

\begin{figure}[H]
    \setlength{\captionmargin}{2pt}
    \centering
    \subfigure[The optimized double torus mesh.]{\includegraphics[scale=0.5]{figures/mesh/doubletorus.pdf}}
    \hspace{1cm}    
    \subfigure[The optimized heart mesh.]{\includegraphics[scale=0.48]{figures/mesh/heart.pdf}}
    \hspace{1cm}
    \subfigure[The optimized orthocircle mesh.]{\includegraphics[scale=0.35]{figures/mesh/orthocircle.pdf}}
    \caption{The optimized meshes on general surfaces.}
    \label{fig:mesh-CGAL}
\end{figure}

\section{SCFT on the general curved surface}
\label{sec:scft}

Our implementation of SCFT on a general curved surface is built on a
standard field theory model for an incompressible AB diblock copolymer melt.
We consider $n$ AB diblock copolymers with the polymerization of $N$
confined on a curved surface with a total surface area $|\calS|$.
The volume fraction of the $A$ block is $f$ and that of the $B$ block is $1-f$.
The two blocks disfavor with each other characterized by the
Flory-Huggins parameter $\chi$. 
We assume that the statistical segment length and segment volume
of the two polymers are equal, i.e., $b_A=b_B=b$ and $v_A=v_B=v_0$.
A characteristic length of the copolymer chain can be defined
by the unperturbed radius of gyration, which is used as the unit
of length so that all spatial lengths are presented in units of
$R_g=b\sqrt{N/6}$. With the incompressible melt assumption, the
average segment density is uniform in space and given by
$v_0=1/\rho_0=V/(n N)$, $\rho_0$ is the number density. Here we use a
continuous Gaussian chain to model the block
copolymer\,\cite{fredrickson2006equilibrium,matsen2002standard}.
By the use of statistical mechanics, 
Hubbard-Stratonovich transformation and infinite dimensional
Fourier transformation, one can derive the
field-based model to describe the phase behavior of the block
copolymer system. However, directly solving the field-based model
is nontrivial due to the very complicated nature of the
functional integral exhibited in the partition
function\,\cite{fredrickson2006equilibrium}. 
In order to simplify the field model we often use an analytic
approximation technique called the mean-field (saddle-point)
approximation which ignores field fluctuations and assumes that
the functional integral is dominated by a single field configuration.
This approximation method become accurate when the dimensionless
chain concentration $C=\rho_0 R_g^3/N$ approaches infinity, and
in the case of high molecular weight block copolymer melts, where
$C$ can be very large. From the viewpoint of mathematics, the
asymptotic method approximate the functional integral (the
partition function) with a large parameter $C$ by the integrand
function on the critical point (actually the saddle point).
Then one can obtain the SCFT equations.
Specifically, within the mean-field approximation, the effective
Hamiltonian of this system is
\begin{align}
  H[w_+, w_-] = &
  \frac{1}{|S|}\int d x\, \left\{ 
  -w_+(x) + \frac{w_-^2(x)}{\chi N} 
  \right\} -\log Q[w_+(x), w_-(\bx)]..
  \label{eq:hamiltonian}
\end{align}
In the above equation, 
%$\chi$ is the Flory-Huggins parameter to
%describe the interaction between segments $A$ and $B$. 
$w_+(\bx)$ and $w_-(\bx)$ are the fluctuating pressure and exchange
chemical potential fields, respectively. The pressure field
enforces the local incompressibility, while the exchange chemical
potential is conjugate to the density operators.
$Q$ is the single chain partition 
functional subjected to external fields $w_+$ and $w_-$.
$d\bx$ is the area element of the surface.
First-order variations of the free energy functional with respect
to the fields lead to the following SCFT equations,
\begin{align}
	& \frac{\delta H}{\delta w_+(\bx)} = \phi_A(\bx) + \phi_B(\bx) - 1 = 0,
	\label{eq:scftwplus}
	\\
	& \frac{\delta H}{\delta w_-(\bx)} = 
   \frac{2 w_-(\bx)}{\chi N} - [\phi_A(\bx) - \phi_B(\bx)] = 0.
	\label{eq:scftwminus}
\end{align}
$\phi_A(\bx)$ and $\phi_B(\bx)$ are $A$ and $B$ monomer densities. 
Then by the use of the continuous Gaussian chain model,
we can complete the SCFT system.
\begin{align}
  & Q = \frac{1}{|\calS|}\int d\bx\,q(\bx,1) = \frac{1}{|\calS|}\int
    d\bx\,q(\bx,s)q^\dag(\bx,1-s), ~~~ \forall s \in [0,1]
  \\
  & \phi_A(\bx) =
  \frac{1}{Q}\int_0^f ds\,q(\bx,s)q^{\dag}(\bx,1-s),
  \label{eq:scft:phiA}
  \\
  & \phi_B(\bx) =
  \frac{1}{Q}\int_f^1 ds\,q(\bx,s)q^{\dag}(\bx,1-s).
  \label{eq:scft:phiB}
\end{align}
In the system, $q(\bx,s)$, the forward propagator, represents the probability
weight that the chain of contour length $s$ has its end at surface position
$\bx$, where the variable $s$ is used to parameterize each copolymer chain. $s
= 0$ represents the tail of the A block and $s = f$ is the junction between the
A and B blocks.  From the flexible Gaussian chain model, the forward propagator
$q(\bx,s)$ satisfies the modified diffusion equation (MDE)
\begin{equation}
  \label{eq:MDE}
	\begin{aligned}
  \frac{\partial }{\partial s}q(\bx,s) &=
	[R_g^2 \Delta_{\calS} - w(\bx,s)] q(\bx,s), 
	\\
  q(\bx,0) &= 1,
  \\
	w(\bx,s) &= \left\{   
	\begin{array}{rl}
		w_+(\bx)-w_-(\bx), &  0\leq s \leq f,
		\\
		w_+(\bx) +w_-(\bx), &  f\leq s \leq 1,
	\end{array}
\right.
	\end{aligned}
\end{equation}
The backward propagator $q^{\dag}(\bx,s)$, which represents the probability 
weight from $s=1$ to $s=0$, satisfies similar MDE  
\begin{equation}
  \label{eq:MDEplus}
	\begin{aligned}
  \frac{\partial }{\partial s}q^\dag(\bx,s) &=
	[R_g^2 \Delta_{\calS} - w^\dag(\bx,s)] q(\bx,s), 
	\\
  q^\dag(\bx,0) &= 1,
  \\
	w^\dag(\bx,s) &= \left\{   
	\begin{array}{rl}
		w_+(\bx)+w_-(\bx), &  0\leq s \leq 1-f,
		\\
		w_+(\bx) -w_-(\bx), &  1-f\leq s \leq 1.
	\end{array}
\right.
	\end{aligned}
\end{equation}


\section{Methodology}
\label{sec:method}


%\subsection{Iteration Procedure}
%\label{subsec:iteration}
%

The equilibrium point of the SCFT equations corresponds to the ordered phase,
as well as the saddle point from the viewpoint of mathematics.  Finding 
saddle points of the SCFT requires iteration methods.  The previous study has
been shown that the effective Hamiltonian \eqref{eq:hamiltonian} can reach its
local minima along the exchange chemical field $w_-$ and maxima along the
pressure field $w_+$\,\cite{jiang2015analytic, fredrickson2002field}. It still
lacks rigorous theoretic guarantee, however, the numerical behavior has
demonstrated its effectiveness. Based on this observation, we can use an
alternating direction scheme to obtain the saddle points.  Specifically, we
introduce a fictitious time variable $t$, and at each time step, the
saddle-point search is formally given by 
\begin{align}
	\frac{\partial }{\partial t}w_+(\bx,t) &= \frac{\delta
	H[w_+, w_-]}{\delta w_+(\bx, t)},
	\label{eq:eulerwplus}
	\\
	\frac{\partial }{\partial t}w_-(\bx,t) &= -\frac{\delta
	H[w_+, w_-]}{\delta w_-(\bx, t)}.
	\label{eq:eulerwminus}
\end{align}
Clearly, Eqs.\,\eqref{eq:scftwplus} and \eqref{eq:scftwminus} are
satisfied when Eqs.\,\eqref{eq:eulerwplus} and
\eqref{eq:eulerwminus} are stationary.

Within the standard framework of SCFT, we can use the following 
iteration scheme to find the saddle point (FSP) of $w_\pm$:
\begin{description}
	\item[Step 1] Given initial estimations of fields $w_\pm(\bx, 0)$.
    \item[Step 2] Compute forward and backward propagator operators
		$q(\bx,s)$ and $q^\dag(\bx,s)$ on a general curved
		surface (see Sec.\,\ref{subsec:mde}).
	\item[Step 3] Obtain $Q$, $\phi_A(\bx)$ and $\phi_B(\bx)$ by
		integral equations (see Sec.\,\ref{subsec:otherdetail}), and calculate
        the Hamiltonian $H$.
	\item[Step 4] Update fields $w_+(\bx,t)$ and $w_-(\bx,t)$ by 
		Eqs.\,\eqref{eq:eulerwplus} and \eqref{eq:eulerwminus}
		using some iterative methods (see Sec.\,\ref{subsec:otherdetail}). 
	\item[Step 5] Repeat steps 2-4 until a convergence criterion
		is met.
\end{description}

In the polymeric systems, each self-assembled microphase possesses its
characteristic scale due to the statistical length of the polymer chain.  In
the flat space, the size and shape of the computational domain can
affect the energy value of self-assembled structures\,\cite{matsen1994stable,
jiang2013discovery} and the properties of the polymer materials, such as the
mechanical property\,\cite{barrat2005introducing, tyler2003stress}.  As in the
flat cases, the ordered pattern formed on the curved surface has its
characteristic scale. In contrast, the size of the curved surface also affects the
self-assembled patterns.  For example, on the sphere surface, the radius of
sphere affects the self-assembled phases of diblock copolymers and their energy
values\,\cite{chantawansri2007}.  However, it still lack a systematical
method to select an optimal size for a self-assembled structure on a
general curved surface. Here, we develop a theoretic tool to efficiently handle
this issue, i.e., an adaptive approach of optimizing the size of a general
curved surface for a given ordered structure.

In the special case of SCFT, the effective Hamiltonian shall be viewed as a
function of the size of the curved surface, as well as a functional of the
field functions.  In the practical implementation, we parameterize the surface
mesh $\mathcal{S}$ by 
\begin{align}
	\mathcal{S}_\Gamma=\{\Gamma\cdot\bx: \bx\in\mathcal{S}_0 \},
	\label{}
\end{align}
where $\calS_0$ is the initial surface mesh, $\Gamma>0$ is a scale factor.
%For example, the parameter set $\Gamma$ of a spherical surface is
%the spherical radius. 
%For ellipsoid surface, $\Gamma$ contain three principal semi-axes.
%Then the effective Hamiltonian \eqref{eq:hamiltonian} can be
%viewed as a functional of $H[w_+,w_-,\Gamma]$. 
%Let $\Gamma=\{\gamma_1,\dots,\gamma_k\}$ be the parameter set of
%description of the size of a given curved surface. 
The scale factor shall be optimized coupled with 
the procedure of FSP. From the aspect of mathematics, the
completed optimization problem of solving SCFT is 
\begin{align}
	\min_{\Gamma}\max_{w_+}\min_{w_-} H[w_+(\bx),
    w_-(\bx), \Gamma]. 
	\label{eq:scftOpt}
\end{align}
Therefore, the integrated SCFT iteration procedure, including seeking
the saddle point of SCFT equations and adaptively optimizing
the scale factor of the curved surface, contains the following steps:
\begin{description}
  \item[Step 1] Give proper parameters $\chi$, $f$, a curved
      surface $\calS$, and reasonable initial distributions of $w_{\pm}$.
  \item[Step 2] Fix $\mathcal{S}$, find the
	  saddle-point of SCFT equation by FSP procedure
	  and obtain the effective Hamiltonian.
  \item[Step 3] Fix $w_\pm$, optimize the size of $\mathcal{S}$ by the method described in
	  Sec.\,\ref{subsec:adjustBox}, calculate effective
	  Hamiltonian.
  \item[Step 4] Repeat steps 2-3 until the effective
	  Hamiltonian discrepancy is lower that a convergence
	  criterion.
\end{description}

\subsection{Surface finite element method for MDE}
\label{subsec:mde}


In the iteration scheme outlined above, the most costly step is solving the 
MDEs of \eqref{eq:MDE} and \eqref{eq:MDEplus}.  For a special
surface, such as the sphere surface, the global basis of spherical harmonics
can be used to expand the order parameters. For a general curved surface,
however, the global basis might not exist. Therefore, the methods based on a
local basis shall be developed to solve the curved surface SCFT, such as the
finite volume or element method. Here, we focus our attention on the
development of the finite element method. We discretize the Laplace-Beltrami
operator in \eqref{eq:MDE} and \eqref{eq:MDEplus} by the surface linear finite
element method.  

Here we just present the finite element discretization of \eqref{eq:MDE}, and
for the equation \eqref{eq:MDEplus} of backward propagator $q^{\dag}$, one can
obtain the discretization scheme similarly. Firstly, we rewrite \eqref{eq:MDE} into the variational
formulation: find $q$ in $H^1(\calS)$, which satisfies
\begin{equation}\label{eq:FEM}
    \left(\frac{\partial}{\partial s} q, v\right)_\calS = -(\nabla_\calS q,
    \nabla_\calS v)_\calS - (wq, v)_\calS, ~~\text{ for all } v\in H^1(\calS),
\end{equation}
where $(\cdot,\cdot)_S$ represents the $L^2$ inner product on
$\calS$. 

Replacing the infinite dimensional space $H^1(\calS)$ by the finite dimension space
$\mathcal V_h$, we obtain the linear finite element discretization of
\eqref{eq:FEM}: find $q_h=\sum_{i=1}^N q_i(s)\varphi_i(\bx)$ in
$\mathcal V_h$, which satisfies
\begin{equation}\label{eq:FEM1}
    \left(\frac{\partial}{\partial s} q_h, v_h\right)_{\calS_h} =
    -(\nabla_{\calS_h}
    q_h, \nabla_{\calS_h} v_h)_{\calS_h} - (w_hq_h, v_h)_{\calS_h},
    ~~\text{ for all } v_h\in \mathcal V_h, 
\end{equation}
where $(\cdot, \cdot)_{S_h}$ represents the $L^2$ inner product on $S_h$ and
$w_h(\bx, s)=\sum_{i=1}^N w(\bx_i, s)\varphi_i(\bx)$ is the linear interpolation of
$w(\bx, s)$.  Replacing $v_h$ by $\varphi_j$, $j=1, 2,
\cdots, N$, we can have the matrix form of \eqref{eq:FEM1}, 

\begin{equation}
    M\frac{\partial}{\partial s}\mathbf q(s) = - (A + F) \mathbf q(s), 
\end{equation}
where 
\begin{equation*}
    \mathbf q(s) = \left(q_1(s), q_2(s), \cdots, q_N(s)\right)^T, 
\end{equation*} 
and 
\begin{equation*}
    M_{i,j} = (\varphi_i, \varphi_j), A_{i, j} = (\nabla_\calS\varphi_i,
    \nabla_\calS\varphi_j), F_{i, j} = (w_h\varphi_i, \varphi_j).
\end{equation*}
Next in the $s$-direction, we use the Crank-Nicolson scheme,
\begin{equation}\label{eq:FEMs}
    M \frac{\mathbf q^{n+1} - \mathbf q^n}{d s} = - \frac{1}{2}(A + F)\left[\mathbf
    q^{n+1} + \mathbf q^{n}\right],
\end{equation}
where $d s$ is the time step size. Combining with the above two steps, we obtain the fully discretization
format
\begin{equation}\label{eq:FEMs}
    \left[M + \frac{ds}{2} (A+F)\right] \mathbf q^{n+1} = -
	\frac{ds}{2}(A + F)\mathbf{q}^{n}.
\end{equation}
When the linear algebraic system \eqref{eq:FEMs} is relatively
small, we solve it with the direct method. Otherwise the algebraic
multigrid method is adapted\,\cite{chenlong}.

\subsection{Adaptively optimizing the curved surface}
\label{subsec:adjustBox}

When fixed fields $w_\pm(\bx)=w_\pm^f(\bx)$, the optimization
problem of \eqref{eq:scftOpt} becomes 
\begin{align}
    \min_{\Gamma}H[w_+^f(\bx), w_-^f(\bx), \Gamma].
	\label{eq:scftoptival}
\end{align}
Any appropriate optimization method can be chosen to solve this
issue, such as the simplest steepest descend method, i.e.,
$d\Gamma/dt = -dH/d\Gamma$.
However, due to the complexity of SCFT, it is hardly written
down the derivative of the
SCFT's Hamiltonian with respect to the scale factor of a general
curved surface $d H/d \Gamma$ analytically.
Alternatively, we turn to the numerical approach to approximate
the analytic expression. We can calculate the
derivative of $H$ with respect to $\Gamma$ by the finite
difference method. 
To improve the effectiveness of algorithm, in practical
implementation, we adapt the nonlinear conjugate-gradient
(CG) method to minimize objective
function\,\eqref{eq:scftoptival},\cite{nocedal2006numerical}, i.e.,
\begin{align}
	\Gamma^{k+1} = \Gamma^{k} + \alpha^k d^k,
  \label{eq:adjustR}
\end{align}
where $d^k$ is the conjugate gradient direction in the $k$-step.
$\alpha_k$ is the step size obtained by the linear search
approach. Meanwhile, the restart technology has been used in
nonlinear CG method to avoid the oscillation and improve the
convergent rate.

\subsection{Other computational details}
\label{subsec:otherdetail}

\subsubsection{Iterative method}
\label{subsubsec:euler}

%\textcolor{red}{The following iterative method might be wrong. It
%may be considered the manifold constraint (Ref. optimization on
%Stiefel manifolds).}

The iterative methods to update the fields 
are dependent on the mathematical structure of SCFT. 
An important fact is that the effective
Hamiltonian \eqref{eq:hamiltonian} can reach its local minima
along the exchange chemical field $w_-(\bx)$, and reach the maxima
along the pressure field
$w_+(\bx)$\,\cite{fredrickson2006equilibrium}. 
A simple and efficient method, i.e., the explicit Euler method, can be
chosen to update the fields $w_\pm(\bx)$. Specifically, this
approach is expressed as 
\begin{equation}
	\begin{aligned}
		w_+^{k+1}(\bx) &= w_+^k(\bx) + \lambda_+
		[\phi_A^k(\bx)+\phi_B^k(\bx)-1 ] 
		\\
		w_-^{k+1}(\bx) &= w_-^k(\bx) - \lambda_-
   \left[\frac{2 w_-^k(\bx)}{\chi N} - [\phi_A(\bx) - \phi_B(\bx)]
   \right].
	\end{aligned}
	\label{}
\end{equation}

\subsubsection{Integral formula along $s$-direction }
\label{subsubsec:sintegral}

A modified numerical integration formula for closed interval is
chosen to evaluate the integrated equations
\eqref{eq:scft:phiA}-\eqref{eq:scft:phiB} that can guarantee
fourth-order precision in $s$-direction whether the number of
discretization points is even or odd\,\cite{press1992numerical}.
\begin{align}
	\int_{0}^{n_s} dt\,f(s) = \Delta s \left\{
	-\frac{5}{8}(f_0+f_{n_s}) 
	+\frac{1}{6}(f_1+f_{n_s -1}) 
	-\frac{1}{24}(f_2+f_{n_s-2}) 
	+\sum_{j=0}^{n_s} f_{j} 
	\right\}.
\end{align}

\subsubsection{Surface integral $\int d\bx$ }

Given a linear finite element function $v_h = \sum_{i=1}^N v_i\varphi_i(\bx)$ on
$\calS_h$. Notice that $\varphi_i(\bx)$ is a linear function about $\bx$ on each
triangle faces, and from \eqref{eq:varphi}, we can compute the integration of $v_h$ on
$\calS_h$ as follows: 

\begin{equation*}
    \int_{\calS_h} v_h \mathrm d \bx = \sum_{\tau_h\in\mathcal T_h} \int_{\tau_h}
    v_i\varphi_i(\bx) + v_j\varphi_j(\bx) + v_k\varphi_k(\bx) \mathrm d \bx = \sum_{\tau_h\in\mathcal T_h} \frac{v_i + v_j +
    v_k}{3}|\tau_h|, 
\end{equation*}
where $v_i$, $v_j$ and $v_k$ are the function values on the three vertices
of triangle face $\tau_h = (\bx_i, \bx_j, \bx_k)$, $|\tau_h|$ is the area of
$\tau_h$.

\section{Numerical Results}
\label{sec:rslt}

In this section, we will use five closed curved surfaces including sphere,
torus, double torus, heart, orthocircle, and an open parabolic surface to
demonstrate the efficiency of our numerical method.  To ensure the accuracy, we
set the substep of contour length $s$ to $0.005$ in the following computation.
The convergence criterion of FSP is 
\begin{align}
	\max\left\{
	\left\|\frac{\delta H}{\delta w_+} \right\|_{\ell^\infty},
	\left\|\frac{\delta H}{\delta w_-} \right\|_{\ell^\infty} 
	\right\} \leq 10^{-6},
	\label{}
\end{align}
and SCFT is the change of effective Hamiltonian smaller than $10^{-6}$.

\subsection{Efficiency of the numerical method}

Since there is no analytical solution of SCFT on a general curved surface, we
have to verify the convergence of the numerical solution of our method by
running the simulations on a series of subdivision meshes. Here we take the
12-microdomain spots structure on the shpere with model parameters $\chi
N=30.0$, $f=0.20$, $\Gamma=2.9$ as an example to show the convergence of our
method,  see Tab.\,\ref{tab:refinemesh}. 

\begin{table}[H]
\caption{
The calculated information of SCFT simulations on a series of subdivision meshes.}
  \label{tab:refinemesh}
  \centering
\begin{tabular}{|c|c|c|c|c|c|}
 \hline
 \# of Node  & \# of Elem. & $\Gamma$ & $\Gamma$-diff. & Hamiltonian &
 Ham. diff. \\
 \hline
162  & 320   & -  & - & -&  -  \\ 
 \hline
 642   & 1280  & 3.7071  & 8.071e-1 & -3.172693  & - \\
 \hline
2562  & 5120  & 3.5786  & 1.285e-1 & -3.160542 &  1.215e-2\\
 \hline
10242 & 20480 & 3.5656  & 1.129e-2 & -3.165431 & 4.889e-3\\
 \hline
40962 & 81920 & 3.5632  & 2.399e-3 & -3.166846 & 1.415e-3 \\
 \hline
\end{tabular}
\end{table}

In the series of simulations, the initial fields on each node $(x, y, z)$ is $
w_{-} = \chi N\sin(5\theta)$, $w_{+} = 0 $, where $\theta = \arctan(y/x) \in
[0, 2\pi]$.  The initial surface is a unit sphere with scale factor
$\Gamma=2.9$.  We uniformly refine the initial mesh and the number of nodes and
triangular elements can be found in the first and second columns of
Tab.\,\ref{tab:refinemesh}. From the table, one can find that, along with the
refinement of the mesh,  the Hamiltonian value is indeed convergent, and more
significantly, the scale factor is adaptively optimized and
convergent. The finally convergent morphology can be found in Fig\,\ref{fig:sphere} (a). 

For better observation, as an example, we detail the SCFT iteration information
of the simulation with $10242$ nodes and $20480$ elements in Tab.\,
\ref{tab:refinemesh}.   The results are demonstrated in
Tab.\,\ref{tab:fixedmesh}.  The effective Hamiltonian and the discrepancy of
Hamiltonian respective to iterations are given in Fig.\,\ref{fig:sphere-5}.
It costs $15$ SCFT iterations, and $831$ FSP iterations. The scale
factor $\Gamma$  is optimized from $2.9$ to $3.5656$.  Correspondingly, the
discrepancy of the effective Hamiltonian values is $5.9\times 10^{-2}$ (from
$-3.106$ to $-3.165$). The change amount of effective Hamiltonian is enough to
determine the thermodynamic stability, as well as the phase boundary in the
phase diagram, of self-assembled structures in block
copolymers\,\cite{chantawansri2007, fredrickson2006equilibrium,
matsen1994stable, jiang2013discovery}.  The results are consistent with the
previous theoretical results\,\cite{chantawansri2007}.  However, it should be
noted that, in Ref.\,\cite{chantawansri2007}, the authors
obtained the accurate effective Hamiltonian values through
manually changing the scale factor $\Gamma$ for each time. Here, we 
can optimal $\Gamma$ automatically. 


\begin{table}[H]
  \caption{
  The iteration details of the SCFT simulation to compute the
  12-microdomain phase on the sphere surface used 
  10242 nodes and 20480 elements, see Tab.\,\ref{tab:refinemesh}).
  }
  \label{tab:fixedmesh}
  \centering
  \begin{tabular}{|c|c|c|c|c|c|}
    \hline
	Iter. of SCFT & Iters. of FSP & $\Gamma$  &
	$\Gamma$-Diff. & Hamiltonian & Ham. Diff.  \\
    \hline
	1 & 169 & 2.90 & - & -3.10648180 & -
	\\ \hline
	2 & 361 & 3.372922 & 4.729e-1 & -3.16079157  & 5.431e-2
	\\ \hline
	3 & 73 & 3.656954 & 2.843e-1  & -3.16437826  & 3.587e-3
	\\ \hline
	4 & 50 & 3.530368 & 1.266e-1  & -3.16530810  & 9.298e-4
	\\ \hline
	5 & 46 & 3.580692 & 5.032e-2  & -3.16538583  & 7.773e-5
	\\ \hline
	6 & 44 & 3.559392 & 2.130e-2  & -3.16543307  & 4.723e-5
	\\ \hline
	7 & 14 & 3.568215 & 8.823e-3  & -3.16529309  & 1.399e-4
	\\ \hline
	8 & 17 & 3.564173 & 4.043e-3  & -3.16546705  & 1.739e-4 
	\\ \hline
	9 & 15 & 3.566281 & 2.018e-3  & -3.16540431  & 6.273e-5
	\\ \hline
    10 & 15 & 3.565232 &1.049e-3  & -3.16543986  & 3.554e-5 
	\\ \hline
    11 & 13 & 3.565744 &5.120e-4  &  -3.16542586 & 1.399e-5
	\\ \hline
	12 & 8 & 3.565524 & 2.212e-4  & -3.16542759  & 1.737e-6 
	\\ \hline
    13 & 2 & 3.565564 & 4.027e-5  &  -3.16543001 & 2.422e-6 
	\\ \hline
    14 & 2 & 3.565581 & 1.061e-6  & -3.16543102  & 1.008e-6
    \\\hline
    15 & 2 & 3.565591 & 1.000e-6  & -3.16543136  & 3.377e-7
    \\ \hline    
  \end{tabular}
\end{table}

\begin{figure}[H]
\setlength{\captionmargin}{2pt}
\centering
    \subfigure[]{\includegraphics[scale=0.55]{figures/mesh/ham1-2.pdf}}
    \hspace{0.0cm}
\subfigure[]{\includegraphics[scale=0.55]{figures/mesh/ham1-1.pdf}}
\    
\caption{
        The effective Hamiltonian (a) and the discrepancy of Hamiltonian (b)
        respective to iterations of the simulation to  compute 12-microdomain
        phase on the sphere surface with 10242 nodes and with 20480 elements,
        see Tab.\,\ref{tab:refinemesh}. 
}
		\label{fig:sphere-5}
\end{figure}


\subsection{Patterns on several curved surfaces}
\label{subsec:pattern}

In this subsection, we will apply our numerical method to several different
curved surfaces, including sphere, torus, double torus, heart, orthocircle
and parabolic surfaces.  Since the SCFT is a high nonlinear system and has
multi-solutions, the initial values play a key role in determining the final
morphologies and accelerating convergent speed.  Starting from random initial
conditions usually does not generate the low energy patterns. In order to
obtain these stable solutions, we often seed some structured profile of
fields in the simulation.

\subsubsection{Sphere}
\label{subsubsec:sphere}

We first show the results of SCFT simulations on the sphere surface.  To capture
these self-assembling precisely, we use enough discretization nodes in these
simulations, as given in the last rows in Tab.\,\ref{tab:sphere:radius}.  The
simulation results for the equilibrium ordered structures are represented in
Fig.\,\ref{fig:sphere}.  The initial scale factor $\Gamma$ and optimal 
$\Gamma_{opt}$ for each self-assembled pattern are given in
Tab.\,\ref{tab:sphere:radius}.  

For the asymmetric composition of AB diblock copolymers with $f=0.20$, spotted
patterns are energetically favorable.  To ensure the accuracy, $10242$ nodes
and $20480$ elements have been used to calculate these asymmetric cases.  In
contrast to the case in 2D flat space where only hexagonal pattern is globally
stable, various spotted patterns appear on the sphere surface. As shown in
Fig.\,\ref{fig:sphere} (a), a 12-microdomain spotted phase whose spots locate
at vertices of a regular icosahedron. The initial fields of this structure have
been given in the above subsection.  Then we enlarge the scale factor $\Gamma$
to $11.65$ and use random initial fields.  A pattern with 116-microdomain
spots, as shown in Fig.\,\ref{fig:sphere} (b), appears in the simulation.  It
has hexagonal lattice structure along with a small number of pentagonal
patterns.  Besides Fig.\,\ref{fig:sphere} (a) and (b), many spotted phases with
different number of microdomains have been also discovered in our
simulations. Obviously, the number of microdomains of a spotted phase is
dependent on the size of the sphere.  Tab.\,\ref{tab:sphere:spotNum} gives the
optimal scale factor of sphere and its corresponding number of microdomains
when $\chi N=25.0$ and $f=0.20$. From this table, one can find that, for
spotted patterns, the number of microdomains is linearly dependent on the
sphere surface area.


\begin{figure}[H]
\setlength{\captionmargin}{2pt}
\centering
    \subfigure[]{\includegraphics[scale=0.27]{figures/mesh/sphere5.pdf}}
    \hspace{0.5cm}
\subfigure[]{\includegraphics[scale=0.30]{figures/mesh/sphere1-2.pdf}}
\\
\subfigure[]{\includegraphics[scale=0.30]{figures/mesh/sphere2-1.pdf}}
    \hspace{0.5cm}
\subfigure[]{\includegraphics[scale=0.30]{figures/mesh/sphere2-2.pdf}}
    \hspace{0.5cm}
\subfigure[]{\includegraphics[scale=0.30]{figures/mesh/sphere2-3.pdf}}
\caption{\label{fig:sphere}
The self-assembled patterns on sphere surface obtained through surface linear finite element SCFT simulations.
The initial and optimized scale factors can be found in
Tab.\,\ref{tab:sphere:radius}.
Yellow colors correspond to large A-segment fractions.
(a) 12-microdomain 5-fold symmetric spotted phase on
the sphere when $\chi N=30.0$, $f=0.20$. 
(b). 116-mircodomain spotted pattern where $\chi N=25.0$, $f=0.20$.
(c)-(e) are three striped configurations density
composition profiles when $\chi N=15.0$, $f=0.50$.
(c). A ring-form phase.
(d). A single spiral ribbon.
(e). A semi-ring striped phase.
}
\end{figure}

\begin{table}[H]
	\caption{The initial scale factor $\Gamma$ and optimal $\Gamma_{opt}$ of
    the sphere surface.}
  \label{tab:sphere:radius}
  \centering
\begin{tabular}{|c|c|c|c|c|c|}
 \hline
 & Fig.\,\ref{fig:sphere} (a) & Fig.\,\ref{fig:sphere} (b) & Fig.\,\ref{fig:sphere} (c) & Fig.\,\ref{fig:sphere} (d) & Fig.\,\ref{fig:sphere} (e)
 \\ \hline
$\Gamma$ & 2.9 & 11.65 & 8 & 7.273 & 8
 \\ \hline
 $\Gamma_{opt}$ & 3.565591 & 10.529 & 7.273 & 5.468 & 6.408 
 \\ \hline
\end{tabular}
\end{table}

\begin{table}[H]
\caption{The relationship between the number of microdomains of a
spotted pattern and the optimal scale factors of sphere when $\chi N=25.0$ and
$f=0.20$. }
  \label{tab:sphere:spotNum}
  \centering
\begin{tabular}{|c|c|c|c|c|c|c|c|c|c|}
 \hline
 $\Gamma_{opt}$ & $3.430$ & $4.593$ & $5.136$ & $5.253$ & $7.411$ & $8.384$ & $9.271$ & $9.841$ & $10.53$
 \\ \hline
 \# of Spots & $12$ & $22$ & $28$ & $29$ & $56$ & $72$ & $89$ & $101$ & $116$
 \\ \hline
\end{tabular}
\end{table}

For the symmetric diblock copolymers, i.e., $f=0.50$, a two-component
alternating flat lamellar phase is energetically favorable. On the sphere
surface, however, various striped patterns can be formed dependent on the
initial fields and the sphere size.  From Fig.\,\ref{fig:sphere} (c-e),
classical striped patterns form including ring and spiral phases.  In these
cases, $2560$ nodes and $5120$ elements have been used to calculate these
asymmetric cases.  Among these patterns, Fig.\,\ref{fig:sphere} (c) is composed
of ordinary ring-form ribbons with point defects at only two opposite site on
the sphere surface.  The pattern in Fig.\,\ref{fig:sphere} (d) is a single
spiral ribbon crawling on the sphere surface with two defects.  The initial
fields for both classical striped phase  Fig.\,\ref{fig:sphere} (c) and a
single spiral ribbon Fig.\,\ref{fig:sphere} (d)  is $w_{-} = \chi
N\sin(8\theta+8\phi), w_{+} = 0$, where $\theta=\arctan(y/x)$,
$\phi=\arctan(z/y) \in [0, 2\pi]$, but with different initial scale factor
$\Gamma$ given in Tab.\,\ref{tab:sphere:radius}. It is obvious that the scale
factor dominates the morphologies of striped phases.  Fig.\,\ref{fig:sphere}
(e) is a semi-ring striped phase whose initial fields is chosen randomly.
More abundant spiral patterns on sphere surface have been found in
Ref.\,\cite{li2006self} with a large range of $\Gamma$. The stability of
striped patterns when $\chi N =12.5$ and $f=0.50$ has been presented by
Chantawansri et. al\,\cite{chantawansri2007} when $\Gamma=3.1 \sim 4.9$.
However, due to the complexity of these striped patterns, the more systematical
studies are still lacking. 

\subsubsection{Torus}
\label{subsubsec:torus}

We next show the results of SCFT simulations on the torus surface. For every
simulation, we fix the rate of the major and minor radius of the torus
surface $\gamma=R/r$ and set the initial scale factors $\Gamma_0$ as 1.  The
initial $R_0$, $r_0$ and optimal scale factor $\Gamma_{opt}$ for each ordered
pattern are presented in Tab.\,\ref{tab:radius:torus}.  The converged
self-assembled structures are given in Fig.\,\ref{fig:torus}. 

\begin{table}[H]
\caption{
    The initial $R_0$, $r_0$ and optimal scale factors $\Gamma_{opt}$ for 
different simulations on the torus surface. }
  \label{tab:radius:torus}
  \centering
\begin{tabular}{|c|c|c|c|c|c|c|c|}
 \hline
 & Fig.\,\ref{fig:torus} (a) & Fig.\,\ref{fig:torus} (b)
    & Fig.\,\ref{fig:torus} (c)& Fig.\,\ref{fig:torus}(d) &
    Fig.\,\ref{fig:torus}(e) &   Fig.\,\ref{fig:torus}(f) &
	Fig.\,\ref{fig:torus}(g) 
    \\ \hline
    $\gamma$ & 8 & 2 & 1.389 &2 &2 & 2 & 2
    \\ \hline
    $R_0$     & 8  & 4 & 3.333 & 4 & 4 & 4 & 4
    \\ \hline
    $r_0$     & 1  & 2 & 2.4 & 2 & 2 & 2 & 2  
	\\ \hline
	$\Gamma_{opt}$ & 1.066 & 1.044 & 1.045 & 0.969 & 1.023 & 1.119 & 1.223
    \\ \hline 
\end{tabular}
\end{table}


For asymmetric AB diblock copolymers, the spotted patterns will appear, as
shown in Fig\,\ref{fig:torus} (a)-(c) where $\chi N = 25.0$ and $f=0.20$.
51200 nodes and 102400 elements are used in these calculations.  Using random
initial condition, the spotted phases with different radii can be found in
Fig.\,\ref{fig:torus} (a) and (b).  To test the influence of initial condition
for spotted phases, we use two ways to choose initial condition.  For
Fig.\,\ref{fig:torus} (c),  the first initial
condition is the deterministic configuration $w_{-} = 0.5(\chi
N\cos(3\theta)\cos(3\phi)+\cos(6\phi)), w_{+} = 0$, and the second is
also the random initial condition.  The same spotted patterns surprisingly
appear after calculating. It indicates that the morphologies of spotted phases
of asymmetric diblock copolymers on torus may be mainly dominated by the geometry
of the surface, i.e., radii $R$ and $r$ of torus, and $f$, $\chi N$.


For the symmetric diblock copolymer system, i.e., $f=0.50$, the equilibrium
striped phases are shown in Fig.\,\ref{fig:torus} (d)-(g) where $\chi N =
16.0$.  $64000$ nodes and $128000$ elements are used in these calculations.  It
is different from spotted patterns, the morphologies of striped phase are
dependent on the initial condition, which is chosen as 
\begin{align*}
    w_- =\chi N\sin(n_\phi \phi + n_\theta \theta), ~~~ w_+ = 0,
\end{align*}
where $\phi, \theta\in[0,2\pi]$ are the parameter coordinates of
torus surface.  If we choose the initial condition with $n_\phi=0$, but with
different $n_\theta$, the unconnected stripes are obtained, as
shown in Fig.\,\ref{fig:torus} (d) ($n_\theta = 8$).
If the initial condition is chosen such that $n_\phi \neq 0$, we
can find the equilibrium connected striped patterns with periods
$n_\theta = n_\phi = 6, 7, 8$, as illustrated in the
Fig.\,\ref{fig:torus} (e)-(g). The results are consistent with
previous calculations\,\cite{li2014mean}.


\begin{figure}[H]
\setlength{\captionmargin}{2pt}
\centering
\subfigure[]{\includegraphics[scale=0.26]{figures/mesh/torus-1.pdf}}
    \hspace{0.5cm}
\subfigure[]{\includegraphics[scale=0.26]{figures/mesh/torus-2.pdf}}
    \hspace{0.5cm}
\subfigure[]{\includegraphics[scale=0.26]{figures/mesh/torus-4.pdf}}
\\
\subfigure[]{\includegraphics[scale=0.26]{figures/mesh/torus5.pdf}}
    \hspace{0.5cm}
\subfigure[]{\includegraphics[scale=0.26]{figures/mesh/torus6.pdf}}
    \hspace{0.5cm}
\subfigure[]{\includegraphics[scale=0.26]{figures/mesh/torus7.pdf}}
    \hspace{0.5cm}
\subfigure[]{\includegraphics[scale=0.26]{figures/mesh/torus8.pdf}}
\caption{The self-assembled patterns on torus are
obtained through surface finite element SCFT calculations.
Yellow colors correspond to large A-segment fractions.
(a)-(c) are spotted patterns where $\chi N= 25.0$, $f=0.20$.
(d)-(g) are stripes phase where $\chi N=16.0$, $f=0.50$.
The information of the torus can be 
found in Tab.\,\ref{tab:radius:torus}.  }
\label{fig:torus}
\end{figure}


\subsubsection{Three general closed surfaces}
\label{subsubsec:generalsurface}

To further demonstrate the power of our proposed method, we will run 
the SCFT simulations on the double torus, heart, and orthocircle surfaces in
this subsection. The parameter setting is $\chi N=25.0$ for asymmetric diblock
copolymer system $f=0.20$, and $\chi N = 15.0$ for symmetric case $f=0.50$. As
can be seen from Fig.\,\ref{fig:othersurface}, we have obtained spotted and
striped phases of diblock copolymers on these  closed surfaces. The initial
$\Gamma$ and optimal scale factor $\Gamma_{opt}$ of the three closed surfaces
are given in Tab.\,\ref{tab:general:radius}.  The expressions of initial
surface $\calS_0$ are given by
Eqns.\,\eqref{fig:mesh:doubletorus}-\eqref{fig:mesh:orthocircle}.  Obviously,
the scale factors of these three surfaces have been successfully optimized
dependent on the specific ordered structures during the iteration procedure.
For spotted structures (Fig.\,\ref{fig:othersurface}(a)-(c)), the initial
condition of field functions is chosen randomly.  For symmetric systems, the
striped phases depend on the initial values. On double torus and orthocircle
surfaces, the morphologies of Fig.\,\ref{fig:othersurface} (d) and (f) are
obtained by setting the initial value of $w_+$ and $w_-$ on each mesh node with
coordinates $(x, y, z)$
\begin{align*}
    w_{-} = \chi N\sin(8x), ~~~ w_{+}  = 0.
\end{align*} 

On heart surface, we use the initial value of Fig.\,\ref{fig:othersurface} (e) of 
\begin{align*}
    w_{-} = \chi N\sin(8\theta), ~~~w_{+} = 0,
\end{align*}
where $\theta = \arctan(y/x)$, $\phi = \arctan(z/y)$, and $\theta, \phi \in [0,
2\pi]$, to obtain the striped structure, as shown in
Fig.\,\ref{fig:othersurface}.  The number of nodes (elements) used on double
torus, heart and orthocircle surfaces are $20642$ ($41288$), $172482$
($344960$), and $21964$ ($43952$), respectively.


\begin{figure}[H]
\setlength{\captionmargin}{2pt}
\centering
\subfigure[]{\includegraphics[scale=0.30]{figures/mesh/doubletorus-1.pdf}}
    \hspace{0.5cm}
\subfigure[]{\includegraphics[scale=0.30]{figures/mesh/heart-1.pdf}}
\hspace{0.5cm}
\subfigure[]{\includegraphics[scale=0.30]{figures/mesh/orthocircle-1.pdf}}
\\
\subfigure[]{\includegraphics[scale=0.30]{figures/mesh/doubletorus-2.pdf}}
\hspace{0.5cm}
\subfigure[]{\includegraphics[scale=0.30]{figures/mesh/heart-3.pdf}}
\hspace{0.5cm}
\subfigure[]{\includegraphics[scale=0.30]{figures/mesh/orthocircle-2.pdf}}
\caption{
    The spotted and striped structures of diblock copolymers on double torus,
    heart, and orthocircle surfaces. Yellow colors still correspond to large
    A-segment fractions.  For spotted phases, the parameter setting is $\chi
    N=25.0$ and $f=0.20$, while for striped patterns, $\chi N=15.0$ and
    $f=0.50$.
}
\label{fig:othersurface}
\end{figure}


\begin{table}[H]
	\caption{The initial $\Gamma$ and optimal scale factors $\Gamma_{opt}$ of 
	double torus, heart and orthocircle surfaces used in the
	surface simulation for diblock copolymers.}
  \label{tab:general:radius}
  \centering
\begin{tabular}{|c|c|c|c|c|c|c|}
 \hline
 & \multicolumn{2}{c}{Double torus} \vline & \multicolumn{2}{c}{Heart}
  \vline &\multicolumn{2}{c}{Orthocircle} \vline
 \\ \hline
    & Fig.\,\ref{fig:othersurface} (a) & Fig.\,\ref{fig:othersurface} (d) &
    Fig.\,\ref{fig:othersurface} (b) & Fig.\,\ref{fig:othersurface} (e) &
    Fig.\,\ref{fig:othersurface} (c) & Fig.\,\ref{fig:othersurface} (f)
 \\ \hline
    $\Gamma$ & 12 & 12 & 5 & 5 & 5 & 5 
    \\ \hline
    $\Gamma_{opt}$ &  12.308  & 14.161 & 5.022  & 5.403   & 5.143 & 4.939
 \\ \hline
\end{tabular}
\end{table}


For asymmetric diblock copolymers, the authors in Ref.\,\cite{li2014self} have
found that the elongated spots usually locate at the saddle points of the
surface.  In our simulations, this phenomenon does not appear.  The discrepancy
may be attributed to the discretization precision and the interaction strength.
In Ref.\,\cite{li2014self}, only $1912$ nodes are used in spotted phase
simulations.  However, our calculation uses, at least, $21560$ nodes (10782
elements) to discretize the surface which is enough to ensure the
discretization precision. The second reason may be the interaction strength.
The interaction parameter used here is $\chi N=25.0$ which is stronger than
that in Ref,\,\cite{li2014self} where $\chi N =13.0$.  The large interaction
parameter means strong segregation which contributes to microscopic phase
separation.  



\subsubsection{Parabolic surface }
\label{subsubsec:parabolic}

In this paper, the last geometry used to study the self-assembled of diblock
copolymers is a parabolic surface which is open.  An appropriate
boundary condition is required in solving the MDEs of
\eqref{eq:MDE} and \eqref{eq:MDEplus}. Here, we used the homogeneous Neumann
boundary condition.  The initial values of $w_+$ and $w_-$ on coordinates $(x,
y, z)$ are given by $w_- = \chi N \sin(k z)$, $w_+=0$ both for asymmetric and
symmetric systems.  $8515$ nodes and $16720$ elements are used in the
following simulations. When $k=5$, we obtain the spotted ($\chi N=25.0$,
$f=0.20$) and striped ($\chi N=15.0$, $f=0.50$) phases, as shown in the
Fig.\,\ref{fig:parabolic}.  The initial parabolic surface $\calS_0$ is $x^2+y^2
= 1$.  The initial scale factor $\Gamma$ and optimal $\Gamma_{opt}$ are given
in Tab.\,\ref{tab:parabolic}. 

\begin{table}[!htpb]
    \caption {The initial scale factor $\Gamma$ and optimal $\Gamma_{opt}$,
    in computing the ordered pattern of diblock copolymers on the
	parabolic surfae.}
  \label{tab:parabolic}
  \centering
\begin{tabular}{|c|c|c|}
 \hline
    & Fig.\,\ref{fig:parabolic} (a) & Fig.\,\ref{fig:parabolic} (b) 
 \\ \hline
    $\Gamma$ & 10 & 10 
    \\ \hline
    $\Gamma_{opt}$ & 10.310 & 10.558
    \\ \hline
\end{tabular}
\end{table}


\begin{figure}[H]
\setlength{\captionmargin}{2pt}
\centering
\subfigure[]{\includegraphics[scale=0.35]{figures/mesh/parabolic1.pdf}}
    \hspace{0.5cm}
\subfigure[]{\includegraphics[scale=0.35]{figures/mesh/parabolic2.pdf}}
\caption{\label{fig:parabolic}
The ordered patterns of diblock copolymers on
parabolic surface are obtained through surface finite elemt SCFT
simulations. 
Yellow colors correspond to large A-segment fractions.
(a) Spotted phase when $\chi N = 25.0$, $f = 0.20$.
(b) Striped pattern when $\chi N = 15.0$, $f = 0.50$.
}
\end{figure}

\section{Conclusion and Outlook}
\label{sec:conclusion}

In this paper, we proposed a surface finite element method to study the
self-assembly behaviors of block copolymers based on a SCFT on general curved
surfaces. To demonstrate the ability of the computational
algorithm, we applied it to diblock copolymer systems on several distinct
curved surfaces, including 5 closed surfaces, i.e., sphere, torus, double
torus, heart, and orthocircle, and an open surface of parabolic surface.
Numerical results illustrate that our method can successfully obtain the
ordered structures of block copolymers on these general curved surfaces. At the
same time, the size of a curved surface indeed has been optimized during the
calculations which can capture the characteristic size of a given
self-assembled structure, and give a high accuracy value of effective
Hamiltonian.  

Block copolymers provide a perfect platform to study the self-assembly
behaviors to themselves and related physical, chemical and biology systems.  To
study the phase behaviors of block copolymers precisely, it still requires a
high precision numerical method.  Here we developed a surface linear finite
element method in an uniform refinement way to discretize the SCFT equation on
a general curved surfaces.  To ensure enough precision, we have to use many
nodes and elements to describe the ordered structures. It costs much
CPU time.  In the future, we will further develop this computational approach
in many directions. The high order finite element method, non-uniform mesh, and
adaptive technique will be choose to improve the precision and reduce the CPU
time cost. Another interesting problem is to extend this computational
framework to more real-life polymeric systems, for example rigid polymers. 


\section*{Acknowledgements} This work is supported by National Science
Foundation of China (11771368, 91430213, 91530321). H. Wei is partially
supported by  Hunan Provincial Civil-Military Integration Industrial
Development Project. K.~Jiang is partially supported by the research grand from
Hunan Science Foundation of China (2018JJ2376), and Youth Project Hunan
Provincial Education Department of China (Grant No. 16B257).


\appendix



\begin{thebibliography}{100}

\bibitem{nakahara2003topology}
M. Nakahara, 
\newblock{Geometry, Topology and Physics}.
Taylor, Francis, 2 edition (2003).

\bibitem{wu2004composite}
Y. Y. Wu,  G. S. Cheng, K. Katsov, S. W. Sides, J. F. Wang, J. Tang, G. H. Fredrickson, M. Moskovits, and G. D. Stucky,
Composite mesostructures by nano-confinement,
{Nat. Mater.}
3 (11) (2004) 816-822.

\bibitem{xiang2005}
H. Q. Xiang, K. Shin, T. Kim, S. Moon, T. J. McCarthy,T. P. Russell, 
The influence of confinement and curvature on the morphology of block copolymers,
J. Polym. Sci. Part B 43 (2005) 3377-3383.

\bibitem{yu2006prl}
B. Yu, P. C. Sun, T. H. Chen, Q. H. Jin, D. T. Ding, B. H. Li,
A.-C. Shi, Confinement-induced novel morphologies of block copolymers, Phys. Rev. Lett. 96 (2006) 138306.

\bibitem{charlotte2011interplay}
{C. R. Stewart-Sloan and E. L. Thomas},
Interplay of symmetries of block polymers and confining geometries,
{Eur. Polym. J}.
47 (4) (2011) 630-646.

\bibitem{segalman2005patterning}
R. A. Segalman,
Patterning with block copolymer thin films,
{Mater. Sci. Eng., R}
46 (6) (2005) 191-226.

\bibitem{chantawansri2007}
T. L. Chantwansri, A. W. Bosse, A. Hexemer, H. D. Ceniceros, C. J.
Garcia-Cervera, E. J. Kramer, G. H. Fredrikson,
Self-consistent field theory simulations of block copolymer assembly on a sphere,
{Phys. Rev. E} 75 (3 Pt 1) (2007) 031802.


\bibitem{li2014self}
J. F. Li, J. Fan, H.D. Zhang, F. Qiu, P. Tang, Y.L. Yang,
Self-consistent field theory of block copolymers on a general curved surface,
{Eur. Phys. J. E} 37 (3) (2014) 9973.

\bibitem{li2006self}
J. F. Li, J. Fan, H.D. Zhang, F. Qiu, P. Tang, Y. L. Yang,
Self-assembled pattern formation of block copolymers on the
surface of the sphere using self-consistent field theory,
{Eur. Phys. J. E} 20 (4) (2006) 449-457.

\bibitem{fredrickson2006equilibrium}
G. H. Fredrickson,
{The equilibrium theory of inhomogeneous polymers},
Oxford University Press: New York, (2006).

\bibitem{matsen1994stable}
M. W. Matsen, M. Schick,
Stable and unstable phases of a diblock copolymer melt,
{Phys. Rev. Lett.}
72 (16) (1994) 2660-2663.


\bibitem{jiang2015self}
K. Jiang, J. Zhang, Q. Liang,
Self-assembly of asymmetrically interacting ABC star triblock copolymer melts,
{J. Phys. Chem. B} 119 (45) (2015) 14551-14562.

\bibitem{xu2013strategy}
W. Xu, K. Jiang, P. Zhang, A. C. Shi,
A strategy to explore stable and metastable ordered phases of block copolymers,
{J. Phys. Chem. B} 117 (17) (2013) 5296-5305.

\bibitem{jiang2010spectral}
K. Jiang, Y. Huang, P. Zhang,
Spectral method for exploring patterns of diblock copolymers,
{J. Comput. Phys.} 229 (20) (2010) 7796-7805.

\bibitem{jiang2013discovery}
K. Jiang, C. Wang, Y. Huang, P. Zhang,
Discovery of new metastable patterns in diblock copolymers,
{Commun. Comput. Phys.} 14 (2) (2013) 443-460.

\bibitem{drolet1999combinatorial}
F. Drolet, G. H. Fredrickson,
Combinatorial screening of complex block copolymer assembly
with self-consistent field theory, 
{Phys. Rev. Lett.}
83 (21) (1999) 4317-4320.

\bibitem{guo2008discovering}
Z. Guo, G. Zhang, F. Qiu, H. Zhang, Y. Yang, A. C. Shi,
Discovering ordered phases of block copolymers:
new results from a generic fourier-space approach,
{Phys. Rev. Lett.}
101 (2) (2008) 28301.


\bibitem{rasmussen2002improved}
K. {\O}. Rasmussen, G. Kalosakas,
Improved numerical algorithm for exploring block copolymer mesophases,
{J. Phys.: Condens. Matter}
40 (16) (2002) 1777-1783.

\bibitem{cochran2006stability}
E. W. Cochran, C.J. Garcia-Cervera, G. H. Fredrickson,
Stability of the gyroid phase in diblock copolymers at strong segregation,
{Macromolecules} 39 (7) (2006) 2449-2451.

\bibitem{li2014mean}
D. M. Li, K. W. Liang, T. Gruhn,
{Mean field theory of diblock copolymer on curved manifolds}, 
{Macromolecular Symposia} 346 (1) (2014) 22-31.

\bibitem{vorselaars2011self}
B. Vorselaars, J. U. Kim, T.L. Chantawansri, G.H.
Fredrickson, M.W. Matsen, 
Self-consistent field theory for diblock copolymers grafted to a sphere,
{Soft Matter} 7 (11) (2011) 5128-5137.

\bibitem{Dziuk1988}
{G. Dziuk},
{Finite elements for the Laplace-Beltrami operator on arbitrary surfaces},
{In:  S. Hildebrandt, R. Leis (eds) partial differential equations
and calculus of variations, Lecture Notes in Mathematics, vol 1357. Springer,
Berlin, Heidelbergv}
(1988) 142-155.

\bibitem{Dziuk;Elliott2007}
G. Dziuk and C.M. Elliott, 
Finite elements on evolving surfaces, 
{SIMA J. Numer.  Anal.}
27 (2) (2007) 262-292.


\bibitem{wei2010}
H. Wei, L. Chen, Y. Huang,
Superconvergence and gradient recovery of linear finite
elements for the Laplace-Beltrami operator on general surfaces,
{SIAM J. Numer. Anal.}
48 (2010) 1920-1943.

\bibitem{Dziuk1991}
G. Dziuk, An algorithm for evolutionary surfaces, 
{Numer. Math.} 
58 (1991) 603-611.


\bibitem{Dziuk2007}
G. Dziuk, C. M. Elliott,
Surface finite elements for parabolic equations,
{J. Comput. Math.}
25 (4) (2007) 385-407.

\bibitem{Demlow2009}
{A. Demlow}, 
{Higher-order finite element methods and pointwise error estimates
for elliptic problems on surfaces},
{SIAM J. Numer. Anal.} 47 (2) (2009) 805-827. 

\bibitem{Dziuk;Elliott2013}
{G. Dziuk,  C. M. Elliott},
{Finite element methods for surface PDEs},
{Acta Numerica} 
22 (2017) 289-396. 

\bibitem{rieau2018}
{L. Rineau, M. Yvinec}, 
{3D surface mesh generation}, 
{In CGAL user and reference manual}, {CGAL editorial board, 4.11.1 edition} (2018). 

\bibitem{wei2012}
{H. Wei},
{Finite element superconvergence and mesh generation and optimization in
interface and Laplace-Beltrami problem (Chinese)}, {China: Xiangtan University} (2012).

\bibitem{matsen2002standard}
M. W. Matsen,
The standard gaussian model for block copolymer melts,
{J. Phys.: Condens. Matter} 14 (2002) R21-R47.


\bibitem{jiang2015analytic}
K. Jiang, W. Xu, P. Zhang,
Analytic structure of the SCFT energy functional of multicomponent block
copolymers,
\newblock {Commun. Comput. Phys.} 17 (5) (2015) 1360-1387.

\bibitem{fredrickson2002field}
G. H. Fredrickson, V. Ganesan, F. Drolet,
{Field-theoretic computer simulation methods for polymers and complex fluids},
{Macromolecules} 35 (1) (2002) 16-39.

\bibitem{barrat2005introducing}
J. L. Barrat, G. H. Fredrickson, S. W. Sides,
{Introducing variable cell shape methods in field theory simulations of polymers},
{J. Phys. Chem. B} 109 (14) (2005) 6694-6700.

\bibitem{tyler2003stress}
{C. A. Tyler, D. C. Morse},
{Stress in self-consistent-field theory}, {Macromolecules} 36 (21) (2003) 8184-8188.

\bibitem{chenlong}
{L. Chen},
{Introduce to multigrid methods},
www.math.uci.edu/~chenlong/lectures.


\bibitem{nocedal2006numerical}
J. Nocedal, S. J. Wright, 
{Numerical optimization},
Springer Science (2006).

\bibitem{press1992numerical}
See the formula (4.1.14) on p.\,160
in {Numerical recipes: the art of scientific computing}, 3rd edition,
W. H. Press, A. S. Teukolsky, W. T. Vetterling, B. P. Flannery, Ed. Cambridge
University Press, New York (2007).

%\bibitem{thompson2004improved}
%R. B. Thompson, K. {\O}. Rasmussen, T. Lookman,
%Improved convergence in block copolymer self-consistent field theory by
%anderson mixing,
%{J. Chem. Phys.} 120 (31) (2004) 31-34.

%\bibitem{thompson2004improved}
%R. B. Thompson, K. {\O}. Rasmussen, T. Lookman,
%Improved convergence in block copolymer self-consistent field theory by
%anderson mixing,
%{J. Chem. Phys.} 120 (31) (2004) 31-34.
%
%\bibitem{anderson1965iterative}
%D. G. Anderson,
%Iterative procedures for nonlinear integral equations,
%{J. ACM} 12 (4) (1965) 547-560.
%
%\bibitem{kim2008interaction}
%J. U. Kim, M. W. Matsen,
%Interaction between polymer-grafted particles,
%{Macromolecules} 41 (12) (2008) 4435-4443.
%
%
%\bibitem{nelson1983liquids}
%D. R. Nelson,
%{Liquids and glasses in spaces of incommensurate curvature},
%{Phys. Rev. Lett.} 50 (13) (1983) 982-985.
%
%\bibitem{demlow2007}
%{A. Demlow, G. Dziuk}, 
%{An adaptive finite element method for the Laplace-Beltrami operator on implicitly defined surfaces}, 
%{SIAM J. Numer. Anal.} 45 (1) (2007) 421-442.
%
%\bibitem{Olshanskii;Reusken2014}
%{M. A. Olshanskii, A. Reusken},
%{Error analysis of a space-time finite element method for solving PDEs on
%evolving surfaces}, 
%{SIAM J. Numer. Anal.} 
%52 (4) (2014) 2092-2120.
%

%
%
%\bibitem{ceniceros2004numerical}
%H. D. Ceniceros, G. H. Fredrickson,
%Numerical solution of polymer self-consistent field theory,
%{Multiscale Model. Simul.}
%2 (3) (2004) 452-474.
%


\end{thebibliography}
\end{document}


\endinput
