%==================================================================================================================================%
%========================================================== 总结与展望 ============================================================%
%==================================================================================================================================%
\chapter{总~结~与~展~望}%\label{Chap:Review_and_Prospect}

本文的主要工作是研究的有限区间上的Lévy-Feller对流—扩散方程的初边值问题,利用Grünwald-Letnikov算子和Riemann-Liouville分数阶导数,进过一系列运算与转化,得到离散的有限差分格式,并利用Matlab对算法进行数值模拟,得到了数值解u(x,t)与参数$\theta$,$\alpha$和变量$t$的关系。

本文主要讨论的算例,鉴于其解析解的特殊性,我们只深入探讨了其数值解的影响因素。即对于不同的$\theta$,$\alpha$,$t$,方程的解表现出怎样的形式及特点。

其次,我们数值实验针对的仅仅是特定的Lévy-Feller对流-扩散微分方程,而差分方法在分数阶微分方程上的应用需要我们进一步探讨。

求解此类分数阶偏微分方程,除了文中讨论的差分方法外,刘发旺教授曾首次成功应用了另一种方法——行方法,上述方程的行方法格式为:$$
\frac{d u_{l}}{d t}=-\frac{a}{h^{\alpha}}\left(c_{+} \sum_{k=0}^{l+1}(-1)^{k} \left( \begin{array}{c}{\alpha} \\ {k}\end{array}\right) u_{l+1-k}+c_{-} \sum_{k=0}^{N-l+1}(-1)^{k} \left( \begin{array}{c}{\alpha} \\ {k}\end{array}\right) u_{l-1+k}\right)-b \frac{u_{l}-u_{l-1}}{h}
$$

故我们还可以比较行方法和差分方法所得到的数值解,以及及两种方法对于不同问题或不同条件的有效性和优劣性。




