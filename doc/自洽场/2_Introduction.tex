%==================================================================================================================================%
%============================================================= 第一章 引言 ========================================================%
%==================================================================================================================================%
\pagestyle{plain}
\pagenumbering{arabic}    %设置页码格式为arabic,并把页计数器置为1.
\newpage
\chapter{~引~言}\label{Chap:Introduction}

\section{研究背景及现状}
分数阶微积分作为整数阶微积分理论的推广,其历史可以上溯到莱布尼兹时代,当时它写给洛必达的信中就谈及了$\frac{1}{2}$阶导数的意义及其相关内容。从那时起,Lagrange、Laplace、Fourier、Euler、Riemann、Liouville等人为该领域做出了突出贡献,他们按照不同的基础与目的对分数阶导数做出了多种解释,从而得到了Riemann-Liouville、Grünwald-Letnikov、Caputo、Riesz等多种形式的分数阶微分方程形式。

分数阶微分方程是常被用来描述记忆特性和中间过程,它以分数阶微积分理论为支撑,是传统模型的延伸。在Fractal theory, Random walk, Viscoelastic mechanics等领域中起到了重要的作用,对于对流-扩散问题的数值计算方法的研究至关重要。它既补充微分方程数学理论的空白,又给出了研究Physics、Biology、Chemistry、Economics等范畴更好的数学模型,尤其是随着计算机的普及与成长,各类学者都可以借助程序语言实现算法,描述事物的物理意义,从而促成这些学科的蓬勃发展。

关于分数阶偏微分方程的数值求解,国内外学者提出了许多数值方法,其中有限差分方法发展最为完善。Meescharet和Tadjeran等人首先提出利用一阶经典和移位Grünwald-Letnikov算子;Deng.等人借助WENO格式,提出了空间精度为6阶的有限差分格式;Cui.给出了紧致有限差分格式;Alikhanov于分数阶导数提出了一种新的差分离散公式,并建立了相应的空间四阶和时间二阶的有限差分格式。Liu.等人开创性地提出行方法,采用可变阶、可变步长的向后差分公式,实现了将分数阶偏微分方程到常微分方程系统的转化\cite{lqx2007}。

迄今为止,空间分数阶微分方程大多是含有单侧分数阶导数或者对称的双侧导数,时间分数阶导数一般为Caputo 导数,而空间分数阶导数则有很多种,包括单侧的分数阶导数——Riemann-Liouville分数阶导数、Grünwald-Letnikov分数阶导数、Caputo分数阶导数、Riesz-Feller分数阶导数等\cite{lqx2007}。文章中所研究的Levy-Feller对流-扩散方程就是利用的双侧Riemann-Liouville分数阶导数。

然而,因为分数阶微分方程的特殊性,其数值解法还不很成熟,依旧存在许多难而未决的问题。故如何得到求解分数阶微分方程的可行的数值方法,将会是一项富有价值的研究课题。此亦是本文研究的意义之所在。

本文研究的Lévy-Feller对流-扩散方程归纳于Random walk和一种随机过程的Stable分布,方程中含有非对称的Riesz-Feller分数阶导数,它含有双侧分数阶导数$_{h} D_{ \pm}^{\alpha}$及倾斜度参数$\theta$。

\section{主要工作与安排}
本文的主要工作如下:

文章主要研究Riemann-Liouville导数意义下的Lévy-Feller对流-扩散方程的定解问题。

本文共分四章, 具体安排如下:
第一章,简单地阐述分数阶微积分的历史渊源、研究情况和数值解法的研究价值。

第二章,给出一些预备知识,包括Riemann-Liouville分数阶导数、Grünwald-Letnikov分数阶导数移位算子\cite{lqx2007}\cite{Liu2007Approximation}、Toeplitz矩阵及相关定理。

第三章,研究一维的有限区间上的Lévy-Feller对流-扩散微分方程的有限差分解法,并用Matlab语言编程实现上述算法,利用几个数值试验验证该格式的稳定性及收敛性进行分析,最终讨论其有效性、精确性和可靠性。

第四章,总结本文的主要研究工作及成果,并给出新的研究方向和着手点。
