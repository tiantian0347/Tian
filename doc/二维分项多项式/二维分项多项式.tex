% !Mode:: "TeX:UTF-8"
\documentclass[12pt,a4paper]{article}

%%%%%%%%------------------------------------------------------------------------
%%%% 日常所用宏包

%% 控制页边距
% 如果是beamer文档类, 则不用geometry
\makeatletter
\@ifclassloaded{beamer}{}{\usepackage[top=2.5cm, bottom=2.5cm, left=2.5cm, right=2.5cm]{geometry}}
\makeatother

%% 控制项目列表
\usepackage{enumerate}

%% 多栏显示
\usepackage{multicol}

%% 算法环境
\usepackage{algorithm}  
\usepackage{algorithmic} 
\usepackage{float} 

%% 网址引用
\usepackage{url}

%% 控制矩阵行距
\renewcommand\arraystretch{1.4}

%% hyperref宏包,生成可定位点击的超链接,并且会生成pdf书签
\makeatletter
\@ifclassloaded{beamer}{
\usepackage{hyperref}
\usepackage{ragged2e} % 对齐
}{
\usepackage[%
    pdfstartview=FitH,%
    CJKbookmarks=true,%
    bookmarks=true,%
    bookmarksnumbered=true,%
    bookmarksopen=true,%
    colorlinks=true,%
    citecolor=blue,%
    linkcolor=blue,%
    anchorcolor=green,%
    urlcolor=blue%
]{hyperref}
}
\makeatother



\makeatletter % 如果是 beamer 不需要下面两个包
\@ifclassloaded{beamer}{
\mode<presentation>
{
} 
}{
%% 控制标题
\usepackage{titlesec}
%% 控制目录
\usepackage{titletoc}
}
\makeatother

%% 控制表格样式
\usepackage{booktabs}

%% 控制字体大小
\usepackage{type1cm}

%% 首行缩进,用\noindent取消某段缩进
\usepackage{indentfirst}

%% 支持彩色文本、底色、文本框等
\usepackage{color,xcolor}

%% AMS LaTeX宏包: http://zzg34b.w3.c361.com/package/maths.htm#amssymb
\usepackage{amsmath,amssymb}
%% 多个图形并排
\usepackage{subfig}
%%%% 基本插图方法
%% 图形宏包
\usepackage{graphicx}
\newcommand{\red}[1]{\textcolor{red}{#1}}
\newcommand{\blue}[1]{\structure{#1}}
\newcommand{\brown}[1]{\textcolor{brown}{#1}}
\newcommand{\green}[1]{\textcolor{green}{#1}}


%%%% 基本插图方法结束

%%%% pgf/tikz绘图宏包设置
\usepackage{pgf,tikz}
\usetikzlibrary{shapes,automata,snakes,backgrounds,arrows}
\usetikzlibrary{mindmap}
%% 可以直接在latex文档中使用graphviz/dot语言,
%% 也可以用dot2tex工具将dot文件转换成tex文件再include进来
%% \usepackage[shell,pgf,outputdir={docgraphs/}]{dot2texi}
%%%% pgf/tikz设置结束


\makeatletter % 如果是 beamer 不需要下面两个包
\@ifclassloaded{beamer}{

}{
%%%% fancyhdr设置页眉页脚
%% 页眉页脚宏包
\usepackage{fancyhdr}
%% 页眉页脚风格
\pagestyle{plain}
}

%% 有时会出现\headheight too small的warning
\setlength{\headheight}{15pt}

%% 清空当前页眉页脚的默认设置
%\fancyhf{}
%%%% fancyhdr设置结束


\makeatletter % 对 beamer 要重新设置
\@ifclassloaded{beamer}{

}{
%%%% 设置listings宏包用来粘贴源代码
%% 方便粘贴源代码,部分代码高亮功能
\usepackage{listings}

%% 设置listings宏包的一些全局样式
%% 参考http://hi.baidu.com/shawpinlee/blog/item/9ec431cbae28e41cbe09e6e4.html
\lstset{
showstringspaces=false,              %% 设定是否显示代码之间的空格符号
numbers=left,                        %% 在左边显示行号
numberstyle=\tiny,                   %% 设定行号字体的大小
basicstyle=\footnotesize,                    %% 设定字体大小\tiny, \small, \Large等等
keywordstyle=\color{blue!70}, commentstyle=\color{red!50!green!50!blue!50},
                                     %% 关键字高亮
frame=shadowbox,                     %% 给代码加框
rulesepcolor=\color{red!20!green!20!blue!20},
escapechar=`,                        %% 中文逃逸字符,用于中英混排
xleftmargin=2em,xrightmargin=2em, aboveskip=1em,
breaklines,                          %% 这条命令可以让LaTeX自动将长的代码行换行排版
extendedchars=false                  %% 这一条命令可以解决代码跨页时,章节标题,页眉等汉字不显示的问题
}}
\makeatother
%%%% listings宏包设置结束


%%%% 附录设置
\makeatletter % 对 beamer 要重新设置
\@ifclassloaded{beamer}{

}{
\usepackage[title,titletoc,header]{appendix}
}
\makeatother
%%%% 附录设置结束


%%%% 日常宏包设置结束
%%%%%%%%------------------------------------------------------------------------


%%%%%%%%------------------------------------------------------------------------
%%%% 英文字体设置结束
%% 这里可以加入自己的英文字体设置
%%%%%%%%------------------------------------------------------------------------

%%%%%%%%------------------------------------------------------------------------
%%%% 设置常用字体字号,与MS Word相对应

%% 一号, 1.4倍行距
\newcommand{\yihao}{\fontsize{26pt}{36pt}\selectfont}
%% 二号, 1.25倍行距
\newcommand{\erhao}{\fontsize{22pt}{28pt}\selectfont}
%% 小二, 单倍行距
\newcommand{\xiaoer}{\fontsize{18pt}{18pt}\selectfont}
%% 三号, 1.5倍行距
\newcommand{\sanhao}{\fontsize{16pt}{24pt}\selectfont}
%% 小三, 1.5倍行距
\newcommand{\xiaosan}{\fontsize{15pt}{22pt}\selectfont}
%% 四号, 1.5倍行距
\newcommand{\sihao}{\fontsize{14pt}{21pt}\selectfont}
%% 半四, 1.5倍行距
\newcommand{\bansi}{\fontsize{13pt}{19.5pt}\selectfont}
%% 小四, 1.5倍行距
\newcommand{\xiaosi}{\fontsize{12pt}{18pt}\selectfont}
%% 大五, 单倍行距
\newcommand{\dawu}{\fontsize{11pt}{11pt}\selectfont}
%% 五号, 单倍行距
\newcommand{\wuhao}{\fontsize{10.5pt}{10.5pt}\selectfont}
%%%%%%%%------------------------------------------------------------------------


%% 设定段间距
\setlength{\parskip}{0.5\baselineskip}

%% 设定行距
\linespread{1}


%% 设定正文字体大小
% \renewcommand{\normalsize}{\sihao}

%制作水印
\RequirePackage{draftcopy}
\draftcopyName{XTUMESH}{100}
\draftcopySetGrey{0.90}
\draftcopyPageTransform{40 rotate}
\draftcopyPageX{350}
\draftcopyPageY{80}

%%%% 个性设置结束
%%%%%%%%------------------------------------------------------------------------


%%%%%%%%------------------------------------------------------------------------
%%%% bibtex设置

%% 设定参考文献显示风格
% 下面是几种常见的样式
% * plain: 按字母的顺序排列,比较次序为作者、年度和标题
% * unsrt: 样式同plain,只是按照引用的先后排序
% * alpha: 用作者名首字母+年份后两位作标号,以字母顺序排序
% * abbrv: 类似plain,将月份全拼改为缩写,更显紧凑
% * apalike: 美国心理学学会期刊样式, 引用样式 [Tailper and Zang, 2006]

\makeatletter
\@ifclassloaded{beamer}{
\bibliographystyle{apalike}
}{
\bibliographystyle{unsrt}
}
\makeatother


%%%% bibtex设置结束
%%%%%%%%------------------------------------------------------------------------

%%%%%%%%------------------------------------------------------------------------
%%%% xeCJK相关宏包

\usepackage{xltxtra,fontspec,xunicode}
\usepackage[slantfont, boldfont]{xeCJK} 

\setlength{\parindent}{2em}%中文缩进两个汉字位

%% 针对中文进行断行
\XeTeXlinebreaklocale "zh"             

%% 给予TeX断行一定自由度
\XeTeXlinebreakskip = 0pt plus 1pt minus 0.1pt

%%%% xeCJK设置结束                                       
%%%%%%%%------------------------------------------------------------------------

%%%%%%%%------------------------------------------------------------------------
%%%% xeCJK字体设置

%% 设置中文标点样式,支持quanjiao、banjiao、kaiming等多种方式
\punctstyle{kaiming}                                        
                                                     
%% 设置缺省中文字体
%\setCJKmainfont[BoldFont={Adobe Heiti Std}, ItalicFont={Adobe Kaiti Std}]{Adobe Song Std}   
\setCJKmainfont{SimSun}
%% 设置中文无衬线字体
%\setCJKsansfont[BoldFont={Adobe Heiti Std}]{Adobe Kaiti Std}  
%% 设置等宽字体
%\setCJKmonofont{Adobe Heiti Std}                            

%% 英文衬线字体
\setmainfont{DejaVu Serif}                                  
%% 英文等宽字体
\setmonofont{DejaVu Sans Mono}                              
%% 英文无衬线字体
\setsansfont{DejaVu Sans}                                   

%% 定义新字体
\setCJKfamilyfont{song}{Adobe Song Std}                     
\setCJKfamilyfont{kai}{Adobe Kaiti Std}
\setCJKfamilyfont{hei}{Adobe Heiti Std}
\setCJKfamilyfont{fangsong}{Adobe Fangsong Std}
\setCJKfamilyfont{lisu}{LiSu}
\setCJKfamilyfont{youyuan}{YouYuan}

%% 自定义宋体
\newcommand{\song}{\CJKfamily{song}}                       
%% 自定义楷体
\newcommand{\kai}{\CJKfamily{kai}}                         
%% 自定义黑体
\newcommand{\hei}{\CJKfamily{hei}}                         
%% 自定义仿宋体
\newcommand{\fangsong}{\CJKfamily{fangsong}}               
%% 自定义隶书
\newcommand{\lisu}{\CJKfamily{lisu}}                       
%% 自定义幼圆
\newcommand{\youyuan}{\CJKfamily{youyuan}}                 

%%%% xeCJK字体设置结束
%%%%%%%%------------------------------------------------------------------------

%%%%%%%%------------------------------------------------------------------------
%%%% 一些关于中文文档的重定义
\newcommand{\chntoday}{\number\year\,年\,\number\month\,月\,\number\day\,日}
%% 数学公式定理的重定义

%% 中文破折号,据说来自清华模板
\newcommand{\pozhehao}{\kern0.3ex\rule[0.8ex]{2em}{0.1ex}\kern0.3ex}

\newtheorem{example}{例}                                   
\newtheorem{theorem}{定理}[section]                         
\newtheorem{definition}{定义}
\newtheorem{axiom}{公理}
\newtheorem{property}{性质}
\newtheorem{proposition}{命题}
\newtheorem{lemma}{引理}
\newtheorem{corollary}{推论}
\newtheorem{remark}{注解}
\newtheorem{condition}{条件}
\newtheorem{conclusion}{结论}
\newtheorem{assumption}{假设}

\makeatletter %
\@ifclassloaded{beamer}{

}{
%% 章节等名称重定义
\renewcommand{\contentsname}{目录}     
\renewcommand{\indexname}{索引}
\renewcommand{\listfigurename}{插图目录}
\renewcommand{\listtablename}{表格目录}
\renewcommand{\appendixname}{附录}
\renewcommand{\appendixpagename}{附录}
\renewcommand{\appendixtocname}{附录}
%% 设置chapter、section与subsection的格式
\titleformat{\chapter}{\centering\huge}{第\thechapter{}章}{1em}{\textbf}
\titleformat{\section}{\centering\sihao}{\thesection}{1em}{\textbf}
\titleformat{\subsection}{\xiaosi}{\thesubsection}{1em}{\textbf}
\titleformat{\subsubsection}{\xiaosi}{\thesubsubsection}{1em}{\textbf}

\@ifclassloaded{book}{

}{
\renewcommand{\abstractname}{摘要}
}
}
\makeatother

\renewcommand{\figurename}{图}
\renewcommand{\tablename}{表}

\makeatletter
\@ifclassloaded{book}{
\renewcommand{\bibname}{参考文献}
}{
\renewcommand{\refname}{参考文献} 
}
\makeatother

\floatname{algorithm}{算法}
\renewcommand{\algorithmicrequire}{\textbf{输入:}}
\renewcommand{\algorithmicensure}{\textbf{输出:}}

%%%% 中文重定义结束
%%%%%%%%------------------------------------------------------------------------


\title{二维分片多项式空间}
\author{姓名:田甜}
\date{\chntoday}
\begin{document}
\maketitle
\newpage
\section{二维分片多项式空间}
\subsection{二维线性分片多项式空间}
给定区间 $I = [(x_0,y_0), (x_1,y_1), (x_2,y_2)]$, 其上的线性多项式空间定义如下:
\begin{equation}
P_1(I) = \{v: v(x,y) = a_ix+b_iy+c_i, x,y\in I, a_i, b_i,c_i, \in \mathbb R,i=0,1,2\},
\end{equation}
选择重心坐标函数:

在二维情况下,我们选取三个点$A:(x_0,y_0),B(x_1,y_1),C(x_2,y_2)(A,B,C$不共线)。设它们满足方程\begin{equation}
f_i(x_j,y_j)=\begin{cases}1 \quad i=j
\\ 0\quad i\neq j\end{cases}\qquad i,j=0,1,2
\end{equation}

我们假设\begin{equation}
f_i(x,y)=a_ix+b_iy+c_i
\end{equation}

利用面积法,我们取点$M:(x,y)$,且设$\triangle ABC$为逆时针顺序排列,此时计算$\triangle ABC,\triangle BCM,\triangle AMC,\triangle ABM$的面积$S,S_0,S_1,S_2$有:
\begin{equation*}\begin{aligned}
		S&=\left| \frac{1}{2} \begin{vmatrix}
			x_0&y_0&1\\x_1&y_1&1\\x_2&y_2&1\end{vmatrix}\right|,S_0&=\left| \frac{1}{2} \begin{vmatrix}
			x&y&1\\x_1&y_1&1\\x_2&y_2&1\end{vmatrix}\right|\\
		S_1&=\left| \frac{1}{2} \begin{vmatrix}
			x_0&y_0&1\\x&y&1\\x_2&y_2&1\end{vmatrix}\right|,S_2&=\left| \frac{1}{2} \begin{vmatrix}
			x_0&y_0&1\\x_1&y_1&1\\x&y&1\end{vmatrix}\right|
	\end{aligned}
\end{equation*}
这时可以得到三个基函数为
\begin{equation}
\lambda_i=f_i(x,y)=\frac{s_i}{s},\quad i=0,1,2
\end{equation}
对于 $v \in P_1(I)$, 有
$$
v(x,y) = v(x_0,y_0)\lambda_0 + v(x_1,y_1) \lambda_1+ v(x_2,y_2) \lambda_2
$$
\subsection{$k$ 次多项式空间 $P_k(I)$}

给定区间$I = [(x_0,y_0), (x_1,y_1), (x_2,y_2), (x_3,y_3,z_3)]$, 其上的线性多项式空间定义如下:
\begin{equation}
P_k(I) = \{v: v(x,y) =  \sum_{j=0}^k\sum_{i=0}^j c_{ij}x^iy^{j-i}, x,y\in I, c_{ij} \in \mathbb R\},
\end{equation}

区间 $I$ 上的个 $ k\geq 1 $ 次基函数共有 

$$n_{dof} = k+1,$$

其计算公式如下:

$$
\phi_{m,n,s} = \frac{1}{m!n!s!}\prod_{l_0 = 0}^{m - 1}
(k\lambda_0 - l_0) \prod_{l_1 = 0}^{n-1}(k\lambda_1 -
l_1)\prod_{l_2 = 0}^{s-1}(k\lambda_2 -
l_1).
$$

其中 $ m\geq 0$, $ n\geq 0 $, $ s\geq 0 $, 且 $ m+n+s=k $, 这里规定:

$$
\prod_{l_i=0}^{-1}(k\lambda_i - l_i) := 1,\quad i=0, 1,2
$$

$k$ 次基函数的面向数组的计算
构造向量: 

\[
P = ( \frac{1}{0!},  \frac{1}{1!}, \frac{1}{2!}, \cdots, \frac{1}{k!})
\]

构造矩阵: 

\[
A :=
\begin{pmatrix}
1  &  1  &  1 \\
k\lambda_0 & k\lambda_1& k\lambda_2\\
k\lambda_0 - 1 & k\lambda_1 - 1& k\lambda_2 - 1\\
\vdots & \vdots \\
k\lambda_0 - (k - 1) & k\lambda_1 - (k - 1)& k\lambda_2 - (k - 1)
\end{pmatrix}
\]

对 $A$ 的每一列做累乘运算, 并左乘由 $P$ 形成的对角矩阵, 得矩阵:

\[
B = \mathrm{diag}(P)
\begin{pmatrix}
1 & 1& 1\\
\lambda_0 & \lambda_1& \lambda_2\\
\prod_{l=0}^{1}(k\lambda_0 - l) & \prod_{l=0}^{1}(k\lambda_1 - l) & \prod_{l=0}^{1}(k\lambda_2 - l)\\
\vdots & \vdots  & \vdots\\
\prod_{l=0}^{k-1}(k\lambda_0 - l) & \prod_{l=0}^{k-1}(k\lambda_1 - l) & \prod_{l=0}^{k-1}(k\lambda_2 - l)
\end{pmatrix}
\]

易知, 只需从 $B$ 的每一列中各选择一项相乘(要求二项次数之和为 $k$,
其中取法共有

\[
n_{dof} = {k+1}
\]

构造指标矩阵:
$$
I = \begin{pmatrix}
k  & 0 & 0\\ k-1 & 1 & 0\\ k-1 & 0 & 1\\k-2 & 2 & 0\\ k-2 & 1 & 1\\\vdots & \vdots & \vdots\\1 & 0& k-1\\ 0 & 1& k-1\\\ 0 & 0& k\\

\end{pmatrix}
$$
则第 $i$ 个 $k$ 次基函数可写成如下形式
$$
\phi_i = B_{m,0}B_{n,1}B_{s,2}, 
$$
其中 $ m = I_{i, 0}, n = I_{i, 1}, s = I_{i, 2} $, 并且 $ m + n+s = k$.

$$\nabla \prod_{j=0}^{m-1}\left(k \lambda_{0}-j\right)=k \sum_{j=0}^{m-1} \prod_{0 \leq i \leq m-1, l \neq j}\left(k \lambda_{0}-l\right) \nabla \lambda_{0}$$
$$\nabla \prod_{j=0}^{n-1}\left(k \lambda_{1}-j\right)=k \sum_{j=0}^{n-1} \prod_{0 \leq i \leq n-1, l \neq j}\left(k \lambda_{1}-l\right) \nabla \lambda_{0}$$
$$\nabla \prod_{j=0}^{s-1}\left(k \lambda_{2}-j\right)=k \sum_{j=0}^{s-1} \prod_{0 \leq i \leq s-1, l \neq j}\left(k \lambda_{2}-l\right) \nabla \lambda_{2}$$
\begin{equation*}
	\bf D^0 = 
	\begin{pmatrix}
		k & k\lambda_0 & \cdots & k\lambda_0 \\
		k\lambda_0 - 1 & k & \cdots & k\lambda_0 - 1 \\
		\vdots & \vdots & \ddots & \vdots \\
		k\lambda_0 - (k-1) & k\lambda_0 - (k-1) & \cdots & k 
	\end{pmatrix},
\end{equation*}
\begin{equation*}
	\bf D^1 = 
	\begin{pmatrix}
		k & k \lambda_1 & \cdots & k\lambda_1 \\
		k\lambda_1 - 1 & k & \cdots & k\lambda_1 - 1 \\
		\vdots & \vdots & \ddots & \vdots \\
		k\lambda_1 - (k-1) & k\lambda_1 - (k-1) & \cdots & k 
	\end{pmatrix},
\end{equation*}
\begin{equation*}
	\bf D^2 = 
	\begin{pmatrix}
		k & k\lambda_2 & \cdots & k\lambda_2 \\
		k\lambda_2 - 1 & k & \cdots & k\lambda_2 - 1 \\
		\vdots & \vdots & \ddots & \vdots \\
		k\lambda_2 - (k-1) & k\lambda_2 - (k-1) & \cdots & k 
	\end{pmatrix},
\end{equation*}

把 $\bf D^0$ 、$\bf D^1$和 $\bf D^2$ 的每一列沿行的方向做累乘运算,然后取它们的下三角矩阵,
最后把下三角矩阵的每一行再求和,即可得到矩阵 $\bf B$ 的每一列各个元素的求导后系
数. 可得到矩阵 $\bf D$,其元素定义为 
$$
\bf D_{i,j} = \sum_{m=0}^j\prod_{k=0}^j D^i_{k, m},\quad 0 \le i \le 1,
, 0 \le j \le k-1.
$$
最后,可以用如下的方式来计算 $\bf B$ 的梯度:
\begin{equation*}
	\begin{aligned}
		\nabla \bf B = & \mathrm{diag}(\bf P)
		\begin{pmatrix}
			0 & 0 & 0 \\
			\bf D_{0,0} \nabla \lambda_0 & 
			\bf D_{1,0} \nabla \lambda_1 & 
			\bf D_{2,0} \nabla \lambda_2 \\
			\vdots & \vdots& \vdots\\
			\bf D_{0, k-1} \nabla \lambda_0 &
			\bf D_{1, k-1} \nabla \lambda_1 &
			\bf D_{2, k-1} \nabla \lambda_2
		\end{pmatrix}\\
		= & \mathrm{diag}(\bf P)
		\begin{pmatrix}
			\mathbf 0\\
			\bf D
		\end{pmatrix}
		\begin{pmatrix}
			\nabla \lambda_0 \\
			& \nabla \lambda_1 \\
			&& \nabla \lambda_2
		\end{pmatrix}\\
		= & \bf F 
		\begin{pmatrix}
			\nabla \lambda_0 &  \\
			& \nabla \lambda_1 \\
			&& \nabla \lambda_2
		\end{pmatrix},
	\end{aligned}
\end{equation*}
其中
\begin{equation}\label{eq:F}
\bf F = \mathrm{diag}(\bf P)
\begin{pmatrix} 
\mathbf 0\\ \bf D
\end{pmatrix}.
\end{equation}

\end{document}
